% This is samplepaper.tex, a sample chapter demonstrating the
% LLNCS macro package for Springer Computer Science proceedings;
% Version 2.21 of 2022/01/12
%
\documentclass[runningheads]{llncs}
%
\usepackage[T1]{fontenc}
% T1 fonts will be used to generate the final print and online PDFs,
% so please use T1 fonts in your manuscript whenever possible.
% Other font encondings may result in incorrect characters.
%
\usepackage{hyperref}
\usepackage{graphicx}
\usepackage{quiver}
\usepackage{proof}
\usepackage{tikz-cd}
% \tikzcdset{scale cd/.style={every label/.append style={scale=#1},
%     cells={nodes={scale=#1}}}}
\tikzcdset{row sep/normal=0.22cm}
\tikzcdset{column sep/normal=0.02cm}
% Used for displaying a sample figure. If possible, figure files should
% be included in EPS format.
%
% If you use the hyperref package, please uncomment the following two lines
% to display URLs in blue roman font according to Springer's eBook style:
%\usepackage{color}
%\renewcommand\UrlFont{\color{blue}\rmfamily}
%\urlstyle{rm}
%
\usepackage{amsmath,amssymb,amsfonts}%

%% macros for typesetting
\newcommand{\udl}[1]{\underline{#1}}
\newcommand{\proofbox}[1]{\begin{array}{c} #1 \end{array}}

%% macros for math symbols
\newcommand{\ot}{\otimes}
\newcommand{\cdast}{\circledast}
\newcommand{\Larr}{\Leftarrow}
\newcommand{\Rarr}{\Rightarrow}
\newcommand{\btleft}{\blacktriangleleft}
\newcommand{\btright}{\blacktriangleright}
\newcommand{\sls}{\slash}
\newcommand{\bsls}{\backslash}
\newcommand{\mc}[1]{\mathcal{#1}}
\newcommand{\mf}[1]{\mathsf{#1}}
\newcommand{\vars}[1]{\mf{var} (#1)}
\newcommand{\gs}[1]{\sigma_{X} (#1)}
\newcommand{\GG}{\Gamma}
\newcommand{\Gg}{\gamma}
\newcommand{\GD}{\Delta}
\newcommand{\Gd}{\delta}
\newcommand{\Gl}{\lambda}

%% macros for acronyms 
\newcommand{\MIP}{\textsf{MIP}}
\newcommand{\MIPeq}{{\textsf{MIP}{\circeq}}}
\newcommand{\FL}{$\mathtt{FL}$}
\newcommand{\Lam}{$\mathtt{L}$}
\newcommand{\NL}{$\mathtt{NL}$}

%% macros for derivations
\newcommand{\vd}{\vdash}
\newcommand{\ax}{\mathsf{ax}}
\newcommand{\tl}{\otimes \mathsf{L}}
\newcommand{\tr}{\otimes\mathsf{R}}
\newcommand{\Ll}{{\Larr}\mathsf{L}}
\newcommand{\Lr}{{\Larr}\mathsf{R}}
\newcommand{\Rl}{{\Rarr}\mathsf{L}}
\newcommand{\Rr}{{\Rarr}\mathsf{R}}
\newcommand{\cut}{\mf{cut}}

%% commands for Agda stuff
\newcommand{\At}{\mathsf{At}}
\newcommand{\at}{\mathsf{at}}
\newcommand{\Fma}{\mathsf{Fma}}
\newcommand{\data}{\mathsf{data}}
\newcommand{\Tree}{\mathsf{Tree}}
\newcommand{\Path}{\mathsf{Path}}
\newcommand{\pathT}[1]{\mathsf{Path} ~ #1}
\newcommand{\append}{+\!\!+}
\newcommand{\Sub}{\mathsf{sub}}
\newcommand{\sub}[2]{\mathsf{sub} ~ #1 ~ #2}
\newcommand{\where}{\mathsf{where}}
\newcommand{\Set}{\mathsf{Set}}
\newcommand{\record}{\mathsf{record}}
\newcommand{\field}{\mathsf{field}}
\newcommand{\subst}{\mathsf{subst}}
\newcommand{\Same}{\mathsf{Same}}
\newcommand{\ContainsLeft}{\mathsf{ContainsLeft}}
\newcommand{\ContainsRight}{\mathsf{ContainsRight}}
\newcommand{\LeftRight}{\mathsf{LeftRight}}
\newcommand{\subcases}{\mathsf{SubEqCases}}
\newcommand{\oneeqtwo}{\mathsf{1{\equiv}2}}
\newcommand{\twogtLone}{\mathsf{2{>}L1}}
\newcommand{\twogtRone}{\mathsf{2{>}R1}}
\newcommand{\onegtLtwo}{\mathsf{1{>}L2}}
\newcommand{\onegtRtwo}{\mathsf{1{>}R2}}
\newcommand{\oneLtwoR}{\mathsf{1\sls \bsls 2}}
\newcommand{\oneRtwoL}{\mathsf{2\sls \bsls 1}}
\newcommand{\subeq}{\mathsf{subeq}}
\newcommand{\inT}{\in^{\mf{T}}}


\newcommand{\niccolo}[1]{\textcolor{red}{NV: #1}}
\newcommand{\cheng}[1]{\textcolor{blue}{CSW: #1}}

\begin{document}
%
\title{An Agda Formalization of Nonassociative Lambek Calculus and Its Metatheory}
%
\titlerunning{Agda Formalization of Nonassociative Lambek Calculus}
% If the paper title is too long for the running head, you can set
% an abbreviated paper title here
%
\author{Niccol{\`o} Veltri\orcidID{0000-0002-7230-3436} \and
Cheng-Syuan Wan \orcidID{0000-0003-2053-1688}}
%
\authorrunning{N. Velri and C.-S. Wan}
% First names are abbreviated in the running head.
% If there are more than two authors, 'et al.' is used.
%
\institute{Tallinn University of Technology, Tallinn, Estonia
\\
\email{\{niccolo,cswan\}@cs.ioc.ee}}
%
\maketitle              % typeset the header of the contribution
%
\begin{abstract}
This paper presents a formalization of the nonassociative Lambek calculus in the Agda proof assistant, focusing on three major proof-theoretic properties: cut-elimination, Maehara interpolation, and proof-relevant interpolation. The sequents in nonassociative Lambek calculus with binary trees as their antecedents create subtleties and obscure the process of proving logical properties through case distinctions. We formally characterize these case distinctions for binary trees, which serves as a foundation for proving both cut-elimination and the Maehara interpolation property. Beyond these two properties, we also formalize proof-relevant interpolation and demonstrate that the Maehara interpolation procedure is well-defined with respect to equivalence of derivations. Our work, especially the case distinction type defined here, aims to serve as a foundation for further research and formalization on nonassociative Lambek calculus.
\keywords{Nonassociative Lambek calculus \and Agda \and Cut-elimination \and Maehara interpolation \and Proof-relevant interpolation}
\end{abstract}
%
%
%

\section{Introduction}
From the viewpoint of sequent calculus, substructural logics are defined by the absence of at least one structural rule. A notable instance is Lambek's syntactic calculus \cite{lambek:mathematics:58}, which forbids weakening, contraction and exchange. Its non-associative variant, further disallows associativity.
% also introduced by Lambek \cite{Lambek1961}, disallows associativity as well.
Lambek calculus (both its associative and nonassociative variants) has been studied extensively and has linguistic applications. For a comprehensive introduction, see \cite{moot:categorial:2012}. While Lambek calculus has different presentations, including axiomatic (or syntactic), natural deduction, and sequent calculus, we focus primarily on the proof-theoretic aspect, specifically the sequent calculus formulation.

Sequents in associative Lambek calculus (\Lam) are of the form $\GG \vd C$, where $\GG$ is a \emph{list} of formulae and $C$ is a single formula, while sequents in nonassociative Lambek calculus (\NL) are of the form $T \vd C$ where $T$ is a \emph{binary tree}. This structural difference makes the proofs of logical properties for the \NL\ trickier than for the associative variant.

For example, when proving cut-elimination, in certain cases, we must determine whether the cut formula coincides with the principal formula of the conclusion. For \Lam, the flat antecedents make the situation clear: there are three possibilities—either the cut formula is to the left of the principal formula, to the right, or it is the principal formula itself.

However, for the \NL, while the case distinction follows similar principles, the hierarchical tree structure obscures these relationships. Consider a derivation that ends with the $\tl$ rule, producing a sequent of the form $T[A \ot B] \vd C$. By assumption of the proof of cut-elimination, the antecedent $T[A \ot B]$ admits an alternative representation $T'[D]$ where $D$ is the cut formula. The challenge lies in precisely characterizing the structural relationship between the tensor product $A \ot B$ and the cut formula $D$ within the tree. Unlike in the associative case, this relationship cannot be presented in a left/right positioning way, but we need a more careful presentation of the trees

Though we can intuitively grasp what occurs in cut-elimination proofs, what about properties more complex than cut-elimination? Especially when proving properties related to equivalence of derivations that require nested presentation of trees, we need a more tractable machinery to ensure correctness of proofs.

This need leads us to consider a formal verification of the \NL. Our approach involves paths in trees, allowing each subtree to be represented as a substitution along a path in the original tree. This transforms the problem of determining relative positions between different subtrees into a formalizable question about the relationship between two paths in a tree.
For example, if two formulae $A$ and $B$ are in a tree with paths $p_1$ and $p_2$ respectively, and these paths are disjoint (diverging at some point), then there must exist some common path $q$ and two paths $q_1$ and $q_2$ after divergence such that $p_1$ equals the path concatenation of $q$ and $q_1$, while $p_2$ equals the concatenation of $q$ and $q_2$.

We formalize this case distinction in Agda. Formalization of mathematics and logics has become a trend in both theoretical computer science and mathematics. While there are many suitable proof assistants (e.g., Agda, Coq, HOL4, and Lean), we choose Agda because its code is more mathematical in style and the process of proving theorems resembles providing algorithms to compute desired terms in the target type.
\cheng{Maybe here can be some more explanation about why formalization is important}.

Within proof theory, interpolation properties serve as fundamental property of a logic and of interests by different logicians. The Maehara interpolation property (\MIP), originating from Maehara's proof of Craig interpolation for classical logic \cite{maehara1961}. Maehara's method has been applied and adapted for several logics, especially for substructural logics that admit a cut-free sequent calculus \cite{ono:proof:nonclassical:1998}.

For nonassociative Lambek calculus, traditional approaches to interpolation for \NL\ extensions often focus on establishing properties like the finite model property \cite{buszkowski:2009,buszkowski:2010} and use different constraints on interpolants.
Our method is more akin to traditional Maehara method because our interest lies in a more refined analysis based on equivalence of derivations.
By formalizing both standard Maehara interpolation and its proof-relevant variant (in the sense of \v{C}ubri{\'c} \cite{Cubric1994} and Saurin \cite{Saurin2024}), we aim to capture not only the existence of interpolants but also their relationship to the structure of derivations. This refined approach enables us to address the well-definedness of interpolation procedures with respect to equivalence of derivations, i.e. if we interpolate on two equivalent derivations, then their interpolants are identical and the corresponding derivations are pairly equivalent.

In this paper, we formalize \NL\ and prove properties in Agda: $(i)$ cut-elimination, $(ii)$ Maehara interpolation, $(iii)$ proof-relevant interpolation, and $(iv)$ \linebreak well-definedness of the Maehara interpolation procedure with respect to equivalence of derivations.

The paper proceeds as follows. We begin with the foundations of our formalization, including definitions of trees, paths in trees, substitutions, and case distinctions for identifying relationships between paths. Next, we introduce the sequent calculus of \NL\ and formalize cut-elimination. The subsequent section addresses equivalence of derivations and properties of the $\cut$ rule.
In the final sections, we formally prove Maehara interpolation and proof-relevant interpolation to demonstrate that our formalization provides a correct and usable strategy. It is worth noting that the proof-relevant interpolation results presented here are novel contributions to the study of \NL.

\noindent\textbf{Related formalizations of nonassociative Lambek calculus}
% \\
% Several previous works have formalized aspects of nonassociative Lambek calculus:
\begin{enumerate}
  \item Chapter 4 of Anoun et al.'s tutorial \cite{anoun2004proof} presents a Coq formalization of nonassociative Lambek calculus, covering axiomatic calculus, natural deduction, and sequent calculus.
  \item Tian \cite{tian2017formalized} built a project that ported the Coq proofs above to the HOL4 proof assistant.
  \item Kokke \cite{kokke2017formalising} took a different approach by formalizing Lambek-Grishin calculus in Agda. However, their work focused on axiomatic calculus, where both the antecedent and succedent are single formulae, in contrast to our tree sequent calculus formalization.
\end{enumerate}
While these earlier formalizations primarily focused on implementing the calculi and proving cut-elimination with potential linguistic applications in mind, our project takes a more proof-theoretic perspective and focuses on the equational theory of proofs. We formalize not only cut-elimination but also extend to Maehara interpolation, proof-relevant interpolation, and demonstrate that Maehara interpolation is well-defined with respect to equivalence of derivations.

\section{Formalization Foundations}\label{sec:agda:base}
In this section, we introduce the core elements of our Agda formalization: trees, paths within trees, and substitutions. We focus particularly on the key formalization of determining the relative positions of subtrees of a tree.
It is designed to precisely characterize whether two subtrees with different representations are identical, one contains the other, or they diverge after some common path. This spatial relationship analysis is fundamental to our subsequent formalization of cut-elimination and interpolation properties of \NL.

\noindent\textbf{Formulae, trees and paths.}
In the original paper, formulae are inductively generated by the grammar $A, B ::= X \ | \ A \Larr B \ | \ B \Rarr A \ | \ A \ot B$, where $X$ is drawn from a set $\mathsf{At}$ of atomic formulae.
In our formalization, we postulate a type $\At$ of atomic formulae. We consistently use $X, Y, Z, \ldots$ to represent atomic formulae.
The type $\Fma$ of formulae is defined as the following inductive type:
\[
\begin{array}{rl}
  \multicolumn{2}{l}{\data \:\:  \Fma : \Set \:\: \where} \\
  \;\; \at &: \At \to \Fma \\
  \;\; \_{\Larr}\_ &: \Fma \to \Fma \to \Fma \\ 
  \;\; \_{\Rarr}\_ &: \Fma \to \Fma \to \Fma \\ 
  \;\; \_{\ot}\_ &: \Fma \to \Fma \to \Fma \\
\end{array}
\]
Note that underscores indicate infix operators, so $A \ot B$ represents a formula for any $A,B : \Fma$.

In \cite{moot:categorial:2012}, trees are defined inductively by the grammar $T ::= \Fma \mid (T, T)$.
A context is defined as a tree with a hole, represented recursively as $\mathcal{C} ::= [\bullet] \mid (\mathcal{C}, T) \mid (T, \mathcal{C})$. 

We formalize the type $\Tree$ of trees as the following inductive type:
\[
\begin{array}{rl}
  \multicolumn{2}{l}{\data \:\:  \Tree : \Set \:\: \where} \\
  \;\; \bullet &: \Tree \\
  \;\; \eta &: \Fma \to \Tree \\
  \;\; \_{\cdast}\_ &: \Tree \to \Tree \to \Tree \\
\end{array}
\]

Our definition more closely resembles the definition of contexts in \cite{moot:categorial:2012}. This approach facilitates smoother proofs and allows us to define derivations using the same type. This definition does not introduce inconsistent derivations since the axiom sequents are defined with non-empty antecedents (see Section \ref{sec:calculus} for the formal sequent calculus).


The type $\Path$ of paths of trees is the following inductive type:
\[
\begin{array}{rl}
  \multicolumn{2}{l}{\data \:\:  \Path : \Tree \to \Set \:\: \where} \\
  \;\; \bullet &: \pathT{\bullet} \\
  \;\; \_\btleft\_ &: \forall ~ \{T\} ~ (p : \pathT{T}) ~ U \to \pathT{(T \cdast U)} \\
  \;\; \_\btright\_ &: \forall ~ T ~ \{U\} ~ (p : \pathT{U}) \to \pathT{(T \cdast U)}
\end{array}
\]
Curly brackets are used in Agda to denote implicit arguments.

\noindent\textbf{Substitution and equality of trees.}
The substitution of a tree into a hole is defined recursively:
\begin{displaymath}
  \begin{array}{rcl}
  subst([\bullet], U) &=& U
  \\
  subst((\mc{C},V), U) &=& (subst(\mc{C},U),V)
  \\
  subst((V,\mc{C}), U) &=& (V , subst(\mc{C},U))
  \end{array}
\end{displaymath}
We use $T[\bullet]$ to denote a context and $T[U]$ to abbreviate $subst(T[\bullet], U)$.

In Agda, substitution function is constructed by pattern-matching (i.e. structural recursion) on the path to a specific hole of a tree.
\[
\begin{array}{ll}
  \multicolumn{2}{l}{\Sub : \forall ~ \{T\} \to \pathT{T} \to \Tree \to \Tree}
  \\[2pt]
  \sub{\bullet}{U} &= U
  \\
  \sub{p \btleft V}{U} &= \sub{p}{U} \cdast V
  \\
  \sub{V \btright p}{U} &= V \cdast \sub{p}{U} 
\end{array}
\]

\begin{example}\label{example:tree:and:path}
The following tree is encoded as $(\eta X ~ \cdast ~ \bullet) \cdast ~ \eta Y$ in Agda.
The path to the hole is expressed as $(\eta X \btright \bullet) \btleft \eta Y$ which indicates that starting from the root node, we traverse one step left followed by one step right to reach the hole.
\[
  % https://q.uiver.app/#q=WzAsNSxbMCw0LCJcXGV0YSBYIl0sWzEsMiwiXFxjZGFzdCJdLFsyLDQsIlxcY2RvdCJdLFsyLDAsIlxcY2Rhc3QiXSxbMywyLCJcXGV0YSBZIl0sWzEsMCwiIiwwLHsic3R5bGUiOnsiaGVhZCI6eyJuYW1lIjoibm9uZSJ9fX1dLFsxLDIsIiIsMix7InN0eWxlIjp7ImhlYWQiOnsibmFtZSI6Im5vbmUifX19XSxbMywxLCIiLDIseyJzdHlsZSI6eyJoZWFkIjp7Im5hbWUiOiJub25lIn19fV0sWzMsNCwiIiwwLHsic3R5bGUiOnsiaGVhZCI6eyJuYW1lIjoibm9uZSJ9fX1dXQ==
\begin{tikzcd}
% [sep=tiny]
	&& \cdast \\
	\\
	& \cdast && {\eta Y} \\
	\\
	{\eta X} && \bullet
	\arrow[no head, from=1-3, to=3-2]
	\arrow[no head, from=1-3, to=3-4]
	\arrow[no head, from=3-2, to=5-1]
	\arrow[no head, from=3-2, to=5-3]
\end{tikzcd}
\]
For any tree $U$, the substitution operation that replaces $\bullet$ with $U$ follows the path specified above.
\end{example}
Given two paths $p : \pathT{T}$ and $q : \pathT{U}$, we write $p \append ~ q : \pathT{\sub{p}{U}}$ for the concatenation.
Notice that the concatenation of the two paths is a path in the new tree $\sub{p}{U}$.
\begin{example}\label{example:path:concatenation}
  Recall the tree and path in Example \ref{example:tree:and:path}. We call them $T$ and $p$, respectively.
  Given another tree $U = \eta Z \cdast \bullet$ with the path $q = \eta Z \btright \bullet$, the path $p \append ~ q = (\eta X \btright (\eta Z \btright \bullet)) \btleft \eta Y$, which is a path in the tree $\sub{p}{U}$.
\end{example}
When proving either cut-elimination or interpolation for nonassociative Lambek calculus, a key step is determining different presentations of the same tree. For cut-elimination, this involves finding the relationship between the principal formula of the endsequent and the cut formula. For interpolation, it concerns the relationship with the interpolating tree. 
\begin{example}\label{example:same:tree:diff:sub}
  Considering the tree in Example \ref{example:tree:and:path}, we can represent it in at least two equivalent ways: $\sub{((\eta X \cdast \bullet) \btright \bullet)}{\eta Y}$ or $\sub{(\bullet \btleft \eta Y)}{(\eta X \cdast \bullet)}$.
\end{example}
We formalize this common process by constructing record types that encompass all possible scenarios where the same tree can be presented in two different ways.
Given two paths $p_1 : \pathT{T_1}$ and $p_2 : \pathT{T_2}$ and two trees $U_1$ and $U_2$, if $\sub{p_1}{U_1}$ is equal to $\sub{p_2}{T_2}$, then there are four possibilities:
\\
\udl{$U_1$ is equal to $U_2$.}
In this case, if $\sub{p_1}{U_1}$ is equal to $\sub{p_2}{T_2}$, then $p_1$ and $p_2$ are forced to be equal as well as $T_1$ and $T_2$.
\[
\begin{array}{rl}
  \multicolumn{2}{l}{\record \:\:  \Same ~ (U_1 ~ U_2 : \Tree) : \Set \:\: \where} \\
  \multicolumn{2}{l}{\quad \mathsf{constructor} ~ \mathsf{same}} \\
  \multicolumn{2}{l}{\quad \field} \\
  \;\; \quad eqT &: T_1 \equiv T_2 \\
  \;\; \quad eqU &: U_1 \equiv U_2 \\
  \;\; \quad eqp &: \subst ~ Path ~ eqT ~ p_1 \equiv p_1
\end{array}
\]
Notice that given $x,y : A$, their propositional equality is denoted as $x \equiv y$.
Elements in this type consist of three equalities about original trees, trees used for substitution into holes, and paths, respectively.
\\
\udl{$U_1$ contains $U_2$ in its left subtree.}
In this case, there exists a path in the left subtree of $U_1$ that points to where $U_2$ should be substituted. Put simply, $p_2$ equals $p_1$ extended with the path in the left subtree of $U_1$ without $U_2$ (for brevity, when we say ``left subtree of $U_1$'', we mean the left subtree of $U_1$ minus $U_2$).
Let $W_1$ and $W_2$ be the left and right subtrees of $U_1$, respectively.
Consider trees $T_1$ and $\sub{p_1}{U_1}$ with the following structure (where $V_1$ is an arbitrary tree):

\[
\arraycolsep=1.5cm
\begin{array}{cc}
  % https://q.uiver.app/#q=WzAsMyxbMSwwLCJcXGNkYXN0Il0sWzAsMiwiXFxidWxsZXQiXSxbMiwyLCJWXzEiXSxbMCwxLCIiLDAseyJzdHlsZSI6eyJoZWFkIjp7Im5hbWUiOiJub25lIn19fV0sWzAsMSwicF8xIiwyLHsib2Zmc2V0IjoxLCJjb2xvdXIiOlswLDYwLDYwXSwic3R5bGUiOnsiaGVhZCI6eyJuYW1lIjoibm9uZSJ9fX0sWzAsNjAsNjAsMV1dLFswLDIsIiIsMix7InN0eWxlIjp7ImhlYWQiOnsibmFtZSI6Im5vbmUifX19XV0=
\begin{tikzcd}
	& \cdast \\
	\\
	\bullet && {V_1}
	\arrow[no head, from=1-2, to=3-1]
	\arrow["{p_1}"', shift right, color={rgb,255:red,214;green,92;blue,92}, no head, from=1-2, to=3-1]
	\arrow[no head, from=1-2, to=3-3]
\end{tikzcd}
&
% https://q.uiver.app/#q=WzAsNSxbMiwwLCJcXGNkYXN0Il0sWzEsMiwiXFxjZGFzdCJdLFszLDIsIlZfMSJdLFswLDQsIldfMSJdLFsyLDQsIldfMiJdLFswLDEsIiIsMCx7InN0eWxlIjp7ImhlYWQiOnsibmFtZSI6Im5vbmUifX19XSxbMCwxLCJwXzEiLDIseyJvZmZzZXQiOjEsImNvbG91ciI6WzAsNjAsNjBdLCJzdHlsZSI6eyJoZWFkIjp7Im5hbWUiOiJub25lIn19fSxbMCw2MCw2MCwxXV0sWzAsMiwiIiwyLHsic3R5bGUiOnsiaGVhZCI6eyJuYW1lIjoibm9uZSJ9fX1dLFsxLDMsIiIsMix7InN0eWxlIjp7ImhlYWQiOnsibmFtZSI6Im5vbmUifX19XSxbMSw0LCIiLDIseyJzdHlsZSI6eyJoZWFkIjp7Im5hbWUiOiJub25lIn19fV1d
\begin{tikzcd}
	&& \cdast \\
	\\
	& \cdast && {V_1} \\
	\\
	{W_1} && {W_2}
	\arrow[no head, from=1-3, to=3-2]
	\arrow["{p_1}"', shift right, color={rgb,255:red,214;green,92;blue,92}, no head, from=1-3, to=3-2]
	\arrow[no head, from=1-3, to=3-4]
	\arrow[no head, from=3-2, to=5-1]
	\arrow[no head, from=3-2, to=5-3]
\end{tikzcd}
\\
(T_1)
&
(\sub{p_1}{U_1})
\end{array}
\]
Since $U_1$ contains $U_2$ in its structure, we can diagram $\sub{p_2}{T_2}$ as shown in the left diagram below, where the blue path represents $p_2$. The right diagram illustrates how $p_2$ extends $p_1$. We denote the path in $W_1$ as $q$
\begin{equation}\label{eq:ContainsLeft:p1p2}
  \arraycolsep=1cm
\begin{array}{cc}
% https://q.uiver.app/#q=WzAsNyxbMCw2LCJVXzIiXSxbMSw0LCJXXzEiXSxbMiw2LCJWXzIiXSxbMiwyLCJcXGNkYXN0Il0sWzMsNCwiV18yIl0sWzMsMCwiXFxjZGFzdCJdLFs0LDIsIlZfMSJdLFsxLDAsIiIsMCx7InN0eWxlIjp7ImhlYWQiOnsibmFtZSI6Im5vbmUifX19XSxbMSwyLCIiLDIseyJzdHlsZSI6eyJoZWFkIjp7Im5hbWUiOiJub25lIn19fV0sWzMsMSwiIiwyLHsic3R5bGUiOnsiaGVhZCI6eyJuYW1lIjoibm9uZSJ9fX1dLFszLDQsIiIsMCx7InN0eWxlIjp7ImhlYWQiOnsibmFtZSI6Im5vbmUifX19XSxbNSwzLCIiLDAseyJzdHlsZSI6eyJoZWFkIjp7Im5hbWUiOiJub25lIn19fV0sWzUsNiwiIiwyLHsic3R5bGUiOnsiaGVhZCI6eyJuYW1lIjoibm9uZSJ9fX1dLFs1LDMsIiIsMCx7Im9mZnNldCI6MSwiY29sb3VyIjpbMjQwLDYwLDYwXSwic3R5bGUiOnsiaGVhZCI6eyJuYW1lIjoibm9uZSJ9fX1dLFszLDEsIiIsMCx7Im9mZnNldCI6MSwiY29sb3VyIjpbMjQwLDYwLDYwXSwic3R5bGUiOnsiaGVhZCI6eyJuYW1lIjoibm9uZSJ9fX1dLFsxLDAsIiIsMCx7Im9mZnNldCI6MSwiY29sb3VyIjpbMjQwLDYwLDYwXSwic3R5bGUiOnsiaGVhZCI6eyJuYW1lIjoibm9uZSJ9fX1dXQ==
\begin{tikzcd}[row sep=0.16cm]
	&&& \cdast \\
	\\
	&& \cdast && {V_1} \\
	\\
	& {W_1} && {W_2} \\
	\\
	{U_2} && {V_2}
	\arrow[no head, from=1-4, to=3-3]
	\arrow[shift right, draw={rgb,255:red,92;green,92;blue,214}, no head, from=1-4, to=3-3]
	\arrow[no head, from=1-4, to=3-5]
	\arrow[no head, from=3-3, to=5-2]
	\arrow[shift right, draw={rgb,255:red,92;green,92;blue,214}, no head, from=3-3, to=5-2]
	\arrow[no head, from=3-3, to=5-4]
	\arrow[no head, from=5-2, to=7-1]
	\arrow[shift right, color={rgb,255:red,92;green,92;blue,214}, no head, from=5-2, to=7-1]
	\arrow[no head, from=5-2, to=7-3]
\end{tikzcd}
&
% https://q.uiver.app/#q=WzAsNyxbMCw2LCJVXzIiXSxbMSw0LCJXXzEiXSxbMiw2LCJWXzIiXSxbMiwyLCJcXGNkYXN0Il0sWzMsNCwiV18yIl0sWzMsMCwiXFxjZGFzdCJdLFs0LDIsIlZfMSJdLFsxLDAsIiIsMCx7InN0eWxlIjp7ImhlYWQiOnsibmFtZSI6Im5vbmUifX19XSxbMSwyLCIiLDIseyJzdHlsZSI6eyJoZWFkIjp7Im5hbWUiOiJub25lIn19fV0sWzMsMSwiIiwyLHsic3R5bGUiOnsiaGVhZCI6eyJuYW1lIjoibm9uZSJ9fX1dLFszLDQsIiIsMCx7InN0eWxlIjp7ImhlYWQiOnsibmFtZSI6Im5vbmUifX19XSxbNSwzLCIiLDAseyJzdHlsZSI6eyJoZWFkIjp7Im5hbWUiOiJub25lIn19fV0sWzUsNiwiIiwyLHsic3R5bGUiOnsiaGVhZCI6eyJuYW1lIjoibm9uZSJ9fX1dLFszLDEsIiIsMCx7Im9mZnNldCI6MSwiY29sb3VyIjpbMjQwLDYwLDYwXSwic3R5bGUiOnsiaGVhZCI6eyJuYW1lIjoibm9uZSJ9fX1dLFs1LDMsInBfMSIsMix7Im9mZnNldCI6MSwiY29sb3VyIjpbMCw2MCw2MF0sInN0eWxlIjp7ImhlYWQiOnsibmFtZSI6Im5vbmUifX19LFswLDYwLDYwLDFdXSxbMSwwLCJxIiwyLHsib2Zmc2V0IjoxLCJjb2xvdXIiOlswLDYwLDYwXSwic3R5bGUiOnsiaGVhZCI6eyJuYW1lIjoibm9uZSJ9fX0sWzAsNjAsNjAsMV1dXQ==
\begin{tikzcd}[row sep=0.16cm]
	&&& \cdast \\
	\\
	&& \cdast && {V_1} \\
	\\
	& {W_1} && {W_2} \\
	\\
	{U_2} && {V_2}
	\arrow[no head, from=1-4, to=3-3]
	\arrow["{p_1}"', shift right, color={rgb,255:red,214;green,92;blue,92}, no head, from=1-4, to=3-3]
	\arrow[no head, from=1-4, to=3-5]
	\arrow[no head, from=3-3, to=5-2]
	\arrow[shift right, draw={rgb,255:red,92;green,92;blue,214}, no head, from=3-3, to=5-2]
	\arrow[no head, from=3-3, to=5-4]
	\arrow[no head, from=5-2, to=7-1]
	\arrow["q"', shift right, color={rgb,255:red,214;green,92;blue,92}, no head, from=5-2, to=7-1]
	\arrow[no head, from=5-2, to=7-3]
\end{tikzcd}
\end{array}
\end{equation}
The trees $T_2$ and $U_1$ are the following:
\begin{equation}\label{eq:ContainsLeft:T2U1}
\arraycolsep=1cm
\begin{array}{cc}
% https://q.uiver.app/#q=WzAsNyxbMCw2LCJcXGJ1bGxldCJdLFsxLDQsIldfMSJdLFsyLDYsIlZfMiJdLFsyLDIsIlxcY2Rhc3QiXSxbMyw0LCJXXzIiXSxbMywwLCJcXGNkYXN0Il0sWzQsMiwiVl8xIl0sWzEsMCwiIiwwLHsic3R5bGUiOnsiaGVhZCI6eyJuYW1lIjoibm9uZSJ9fX1dLFsxLDIsIiIsMix7InN0eWxlIjp7ImhlYWQiOnsibmFtZSI6Im5vbmUifX19XSxbMywxLCIiLDIseyJzdHlsZSI6eyJoZWFkIjp7Im5hbWUiOiJub25lIn19fV0sWzMsNCwiIiwwLHsic3R5bGUiOnsiaGVhZCI6eyJuYW1lIjoibm9uZSJ9fX1dLFs1LDMsIiIsMCx7InN0eWxlIjp7ImhlYWQiOnsibmFtZSI6Im5vbmUifX19XSxbNSw2LCIiLDIseyJzdHlsZSI6eyJoZWFkIjp7Im5hbWUiOiJub25lIn19fV1d
\begin{tikzcd}[row sep=0.16cm]
	&&& \cdast \\
	\\
	&& \cdast && {V_1} \\
	\\
	& {W_1} && {W_2} \\
	\\
	\bullet && {V_2}
	\arrow[no head, from=1-4, to=3-3]
	\arrow[no head, from=1-4, to=3-5]
	\arrow[no head, from=3-3, to=5-2]
	\arrow[no head, from=3-3, to=5-4]
	\arrow[no head, from=5-2, to=7-1]
	\arrow[no head, from=5-2, to=7-3]
\end{tikzcd}
&
% https://q.uiver.app/#q=WzAsNixbMCw1LCJVXzIiXSxbMSwzLCJXXzEiXSxbMiw1LCJWXzIiXSxbMiwxLCJcXGNkYXN0Il0sWzMsMywiV18yIl0sWzIsMF0sWzEsMCwiIiwwLHsic3R5bGUiOnsiaGVhZCI6eyJuYW1lIjoibm9uZSJ9fX1dLFsxLDIsIiIsMix7InN0eWxlIjp7ImhlYWQiOnsibmFtZSI6Im5vbmUifX19XSxbMywxLCIiLDIseyJzdHlsZSI6eyJoZWFkIjp7Im5hbWUiOiJub25lIn19fV0sWzMsNCwiIiwwLHsic3R5bGUiOnsiaGVhZCI6eyJuYW1lIjoibm9uZSJ9fX1dXQ==
\begin{tikzcd}[row sep=0.16cm]
	&& {} \\
	&& \cdast \\
	\\
	& {W_1} && {W_2} \\
	\\
	{U_2} && {V_2}
	\arrow[no head, from=2-3, to=4-2]
	\arrow[no head, from=2-3, to=4-4]
	\arrow[no head, from=4-2, to=6-1]
	\arrow[no head, from=4-2, to=6-3]
\end{tikzcd}
\\
(T_2)
&
(U_1)
\end{array}  
\end{equation}
These observations are implemented in Agda as the following:
\[
\begin{array}{ll}
  \multicolumn{2}{l}{\record \:\:  \ContainsLeft ~ (U_1 ~ U_2 : \Tree) : \Set \:\: \where} \\
  \multicolumn{2}{l}{\quad \mathsf{constructor} ~ \mathsf{gt}} \\
  \multicolumn{2}{l}{\quad \field} \\
  \;\; \quad \{W_1 ~ W_2 \} &: \Tree \\
  \;\; \quad q   &: \pathT{W_1} \\
  \;\; \quad eqT &: T_2 \equiv \sub{p_1}{(W_1 \cdast W_2)} \\
  \;\; \quad eqU &: U_1 \equiv \sub{(q \btleft W_2)}{U_2} \\
  \;\; \quad eqp &: \subst ~ Path ~ eqT ~ p_2 \equiv p_1 \append ~ (q \btleft W_2)
\end{array}
\]
Elements in this type include two implicit terms $W_1$ and $W_2$ that represent the left and right subtrees of $U_1$, respectively, along with a path $q : \pathT{W_1}$ that indicates how to extend $p_1$. Additionally, there are three equalities concerning $T_2$, $U_1$, and $p_2$.
The first and second equalities correspond to our observations from the left and right diagrams in (\ref{eq:ContainsLeft:T2U1}), respectively.
The third equality reflects what we observed in the right diagram of (\ref{eq:ContainsLeft:p1p2}).
\\
\udl{$U_1$ contains $U_2$ in its right subtree.}
\[
\begin{array}{ll}
  \multicolumn{2}{l}{\record \:\:  \ContainsRight ~ (U_1 ~ U_2 : \Tree) : \Set \:\: \where} \\
  \multicolumn{2}{l}{\quad \mathsf{constructor} ~ \mathsf{gt}} \\
  \multicolumn{2}{l}{\quad \field} \\
  \;\; \quad \{W_1 ~ W_2 \} &: \Tree \\
  \;\; \quad q   &: \pathT{W_2} \\
  \;\; \quad eqT &: T_2 \equiv \sub{p_1}{(W_1 \cdast W_2)} \\
  \;\; \quad eqU &: U_1 \equiv \sub{(W_1 \btright q)}{U_2} \\
  \;\; \quad eqp &: \subst ~ Path ~ eqT ~ p_2 \equiv p_1 \append ~ (W_1 \btright q)
\end{array}
\]
This represents the dual case of what we just examined. Notice how the triangles in the type are flipped compared to the previous case.
\\
\udl{$U_1$ and $U_2$ are disjoint.}
In this case, $p_1$ and $p_2$ diverge after a certain point. They share a common initial path, but then split at a node where they take different directions. Consider the following tree (where $V$ is an arbitrary tree):
\[
% https://q.uiver.app/#q=WzAsNSxbMiwwLCJcXGNkYXN0Il0sWzEsMiwiXFxjZGFzdCJdLFszLDIsIlYiXSxbMCw0LCJcXGJ1bGxldCJdLFsyLDQsIlxcYnVsbGV0Il0sWzAsMSwiIiwwLHsic3R5bGUiOnsiaGVhZCI6eyJuYW1lIjoibm9uZSJ9fX1dLFswLDIsIiIsMix7InN0eWxlIjp7ImhlYWQiOnsibmFtZSI6Im5vbmUifX19XSxbMCwxLCIiLDIseyJvZmZzZXQiOjEsImNvbG91ciI6WzAsNjAsNjBdLCJzdHlsZSI6eyJoZWFkIjp7Im5hbWUiOiJub25lIn19fV0sWzEsMywiIiwyLHsic3R5bGUiOnsiaGVhZCI6eyJuYW1lIjoibm9uZSJ9fX1dLFsxLDQsIiIsMix7InN0eWxlIjp7ImhlYWQiOnsibmFtZSI6Im5vbmUifX19XSxbMSw0LCIiLDIseyJvZmZzZXQiOi0xLCJjb2xvdXIiOlsyNDAsNjAsNjBdLCJzdHlsZSI6eyJoZWFkIjp7Im5hbWUiOiJub25lIn19fV0sWzEsMywiIiwyLHsib2Zmc2V0IjoxLCJjb2xvdXIiOlswLDYwLDYwXSwic3R5bGUiOnsiaGVhZCI6eyJuYW1lIjoibm9uZSJ9fX1dLFswLDEsIiIsMix7Im9mZnNldCI6LTEsImNvbG91ciI6WzI0MCw2MCw2MF0sInN0eWxlIjp7ImhlYWQiOnsibmFtZSI6Im5vbmUifX19XV0=
\begin{tikzcd}
	&& \cdast \\
	\\
	& \cdast && V \\
	\\
	\bullet && \bullet
	\arrow[no head, from=1-3, to=3-2]
	\arrow[shift right, color={rgb,255:red,214;green,92;blue,92}, no head, from=1-3, to=3-2]
	\arrow[shift left, color={rgb,255:red,92;green,92;blue,214}, no head, from=1-3, to=3-2]
	\arrow[no head, from=1-3, to=3-4]
	\arrow[no head, from=3-2, to=5-1]
	\arrow[shift right, draw={rgb,255:red,214;green,92;blue,92}, no head, from=3-2, to=5-1]
	\arrow[no head, from=3-2, to=5-3]
	\arrow[shift left, draw={rgb,255:red,92;green,92;blue,214}, no head, from=3-2, to=5-3]
\end{tikzcd}
\]
The red path represents $p_1$ while the blue path shows $p_2$ Notice how they share a common initial segment. Let us call this shared part $q$ with the remaining segments of $p_1$ and $p_2$ as $q_1$ and $q_2$ respectively. This gives us the following diagram:
\begin{equation}\label{eq:LeftRight:paths}
% https://q.uiver.app/#q=WzAsNSxbMiwwLCJcXGNkYXN0Il0sWzEsMiwiXFxjZGFzdCJdLFszLDIsIlYiXSxbMCw0LCJcXGJ1bGxldCJdLFsyLDQsIlxcYnVsbGV0Il0sWzAsMSwicSIsMix7ImNvbG91ciI6WzI3MCw2MCw2MF0sInN0eWxlIjp7ImhlYWQiOnsibmFtZSI6Im5vbmUifX19LFsyNzAsNjAsNjAsMV1dLFswLDIsIiIsMix7InN0eWxlIjp7ImhlYWQiOnsibmFtZSI6Im5vbmUifX19XSxbMSwzLCIiLDIseyJzdHlsZSI6eyJoZWFkIjp7Im5hbWUiOiJub25lIn19fV0sWzEsNCwiIiwyLHsic3R5bGUiOnsiaGVhZCI6eyJuYW1lIjoibm9uZSJ9fX1dLFsxLDQsInFfMiIsMCx7Im9mZnNldCI6LTEsImNvbG91ciI6WzI0MCw2MCw2MF0sInN0eWxlIjp7ImhlYWQiOnsibmFtZSI6Im5vbmUifX19LFsyNDAsNjAsNjAsMV1dLFsxLDMsInFfMSIsMix7Im9mZnNldCI6MSwiY29sb3VyIjpbMCw2MCw2MF0sInN0eWxlIjp7ImhlYWQiOnsibmFtZSI6Im5vbmUifX19LFswLDYwLDYwLDFdXV0=
\begin{tikzcd}
	&& \cdast \\
	\\
	& \cdast && V \\
	\\
	\bullet && \bullet
	\arrow["q"', color={rgb,255:red,153;green,92;blue,214}, no head, from=1-3, to=3-2]
	\arrow[no head, from=1-3, to=3-4]
	\arrow[no head, from=3-2, to=5-1]
	\arrow["{q_1}"', shift right, color={rgb,255:red,214;green,92;blue,92}, no head, from=3-2, to=5-1]
	\arrow[no head, from=3-2, to=5-3]
	\arrow["{q_2}", shift left, color={rgb,255:red,92;green,92;blue,214}, no head, from=3-2, to=5-3]
\end{tikzcd}  
\end{equation}
We obtain $T_1$ by substituting $U_2$ at the hole along path $q \append ~ q_2$, while similarly, we get $T_2$ by substituting $U_1$ at the hole along path $q \append ~ q_1$.
\begin{equation}\label{eq:LeftRight:trees}
  \arraycolsep=1cm
\begin{array}{cc}
% https://q.uiver.app/#q=WzAsNSxbMiwwLCJcXGNkYXN0Il0sWzEsMiwiXFxjZGFzdCJdLFszLDIsIlYiXSxbMCw0LCJcXGJ1bGxldCJdLFsyLDQsIlVfMiJdLFswLDEsInEiLDIseyJjb2xvdXIiOlsyNzAsNjAsNjBdLCJzdHlsZSI6eyJoZWFkIjp7Im5hbWUiOiJub25lIn19fSxbMjcwLDYwLDYwLDFdXSxbMCwyLCIiLDIseyJzdHlsZSI6eyJoZWFkIjp7Im5hbWUiOiJub25lIn19fV0sWzEsMywiIiwyLHsic3R5bGUiOnsiaGVhZCI6eyJuYW1lIjoibm9uZSJ9fX1dLFsxLDQsIiIsMix7InN0eWxlIjp7ImhlYWQiOnsibmFtZSI6Im5vbmUifX19XSxbMSw0LCJxXzIiLDAseyJvZmZzZXQiOi0xLCJjb2xvdXIiOlsyNDAsNjAsNjBdLCJzdHlsZSI6eyJoZWFkIjp7Im5hbWUiOiJub25lIn19fSxbMjQwLDYwLDYwLDFdXSxbMSwzLCJxXzEiLDIseyJvZmZzZXQiOjEsImNvbG91ciI6WzAsNjAsNjBdLCJzdHlsZSI6eyJoZWFkIjp7Im5hbWUiOiJub25lIn19fSxbMCw2MCw2MCwxXV1d
\begin{tikzcd}
	&& \cdast \\
	\\
	& \cdast && V \\
	\\
	\bullet && {U_2}
	\arrow["q"', color={rgb,255:red,153;green,92;blue,214}, no head, from=1-3, to=3-2]
	\arrow[no head, from=1-3, to=3-4]
	\arrow[no head, from=3-2, to=5-1]
	\arrow["{q_1}"', shift right, color={rgb,255:red,214;green,92;blue,92}, no head, from=3-2, to=5-1]
	\arrow[no head, from=3-2, to=5-3]
	\arrow["{q_2}", shift left, color={rgb,255:red,92;green,92;blue,214}, no head, from=3-2, to=5-3]
\end{tikzcd}
&
% https://q.uiver.app/#q=WzAsNSxbMiwwLCJcXGNkYXN0Il0sWzEsMiwiXFxjZGFzdCJdLFszLDIsIlYiXSxbMCw0LCJVXzEiXSxbMiw0LCJcXGJ1bGxldCJdLFswLDEsInEiLDIseyJjb2xvdXIiOlsyNzAsNjAsNjBdLCJzdHlsZSI6eyJoZWFkIjp7Im5hbWUiOiJub25lIn19fSxbMjcwLDYwLDYwLDFdXSxbMCwyLCIiLDIseyJzdHlsZSI6eyJoZWFkIjp7Im5hbWUiOiJub25lIn19fV0sWzEsMywiIiwyLHsic3R5bGUiOnsiaGVhZCI6eyJuYW1lIjoibm9uZSJ9fX1dLFsxLDQsIiIsMix7InN0eWxlIjp7ImhlYWQiOnsibmFtZSI6Im5vbmUifX19XSxbMSw0LCJxXzIiLDAseyJvZmZzZXQiOi0xLCJjb2xvdXIiOlsyNDAsNjAsNjBdLCJzdHlsZSI6eyJoZWFkIjp7Im5hbWUiOiJub25lIn19fSxbMjQwLDYwLDYwLDFdXSxbMSwzLCJxXzEiLDIseyJvZmZzZXQiOjEsImNvbG91ciI6WzAsNjAsNjBdLCJzdHlsZSI6eyJoZWFkIjp7Im5hbWUiOiJub25lIn19fSxbMCw2MCw2MCwxXV1d
\begin{tikzcd}
	&& \cdast \\
	\\
	& \cdast && V \\
	\\
	{U_1} && \bullet
	\arrow["q"', color={rgb,255:red,153;green,92;blue,214}, no head, from=1-3, to=3-2]
	\arrow[no head, from=1-3, to=3-4]
	\arrow[no head, from=3-2, to=5-1]
	\arrow["{q_1}"', shift right, color={rgb,255:red,214;green,92;blue,92}, no head, from=3-2, to=5-1]
	\arrow[no head, from=3-2, to=5-3]
	\arrow["{q_2}", shift left, color={rgb,255:red,92;green,92;blue,214}, no head, from=3-2, to=5-3]
\end{tikzcd}
\\
(T_1)
&
(T_2)
\end{array}
\end{equation}
We implement the observations in Agda as the following type:
\[
\begin{array}{ll}
  \multicolumn{2}{l}{\record \:\:  \LeftRight ~ (U_1 ~ U_2 : \Tree) : \Set \:\: \where} \\
  \multicolumn{2}{l}{\quad \mathsf{constructor} ~ \mathsf{disj}} \\
  \multicolumn{2}{l}{\quad \field} \\
  \;\; \quad \{W ~ W_1 ~ W_2 \} &: \Tree \\
  \;\; \quad q       &: \pathT{W} \\
  \;\; \quad q_1     &: \pathT{W_1} \\
  \;\; \quad q_2     &: \pathT{W_2} \\
  \;\; \quad eqT_1   &: \sub{q}{(W_1 \cdast \sub{q_2}{U_2})} \equiv T_1 \\
  \;\; \quad eqT_2   &: T_2 \equiv \sub{q}{(\sub{q_1}{U_1} \cdast W_2)} \\
  \;\; \quad eqp_1   &: \subst ~ Path ~ eqT_1 ~ (q \append ~ (q_1 \btleft W_2)) \equiv p_1 \\
  \;\; \quad eqp_2   &: \subst ~ Path ~ eqT_2 ~ p_2 \equiv q \append ~ (W_1 \btright q_2)
\end{array}
\]
Elements of this type include three implicit terms: $W$, $W_1$, and $W_2$. Here, $W$ represents the structure containing both the common path $q$ and the largest subtree beginning at the divergence point of $p_1$ and $p_2$. Meanwhile, $W_1$ represents the subtree where $U_1$ will be substituted, and $W_2$ represents the subtree where $U_2$ will be substituted.
The remaining equalities reflect what we observed in the diagrams shown in (\ref{eq:LeftRight:paths}) and (\ref{eq:LeftRight:trees}).

While our illustration shows $U_1$ positioned to the left of $U_2$, the dual case where $U_2$ appears to the left of $U_1$ is also possible and is covered by this type definition.

After having distinguished the cases above, we define a datatype to collect all possible cases of the relative positions of $U_1$ and $U_2$.
In the types above, we do not care about the order of $U_1$ and $U_2$ and for a better proving experience (using the datatype in the subsequent proofs), we take the order of $U_1$ and $U_2$ into account and develop seven cases.

\[
\begin{array}{rl}
  \multicolumn{2}{l}{\data \:\:  \mathsf{SubEqCases} \:\: (U_1 ~ U_2 : Tree) : \Set \:\: \where} \\
  \;\; \oneeqtwo &: \Same ~ p_1 ~ p_2 ~ U_1 ~ U_2 \to \subcases ~ U_1 ~ U_2 \\
  \;\; \twogtLone &: \ContainsLeft ~ p_1 ~ p_2 ~ U_1 ~ U_2 \to \subcases ~ U_1 ~ U_2 \\
  \;\; \twogtRone &: \ContainsRight ~ p_1 ~ p_2 ~ U_1 ~ U_2 \to \subcases ~ U_1 ~ U_2 \\
  \;\; \onegtLtwo &: \ContainsLeft ~ p_2 ~ p_1 ~ U_2 ~ U_1 \to \subcases ~ U_1 ~ U_2 \\
  \;\; \onegtRtwo &: \ContainsRight ~ p_2 ~ p_1 ~ U_2 ~ U_1 \to \subcases ~ U_1 ~ U_2 \\
  \;\; \oneLtwoR &: \LeftRight ~ p_1 ~ p_2 ~ U_1 ~ U_2 \to \subcases ~ U_1 ~ U_2 \\
  \;\; \oneRtwoL &: \LeftRight ~ p_2 ~ p_1 ~ U_2 ~ U_1 \to \subcases ~ U_1 ~ U_2
\end{array}
\]
The constructor $\onegtLtwo$ means $U_2$ is inside $U_1$ and the path $p_2$ is longer than $p_1$.
The other constructors are read in a similar manner, but one should be careful about the order of arguments. For example, in the constructor $\twogtLone$, the arguments are swapped.

In the final step, we construct a function to examine the correctness of the types we have defined above.
\[
\begin{array}{l}
 \subeq : \forall ~ \{T_1 ~ T_2 \} ~ U_1 ~ U_2 ~ (p_1 : \pathT{T_1}) ~ (p_2 : \pathT{T_2})
  \\
  \quad \to \sub{p_1}{U_1} \equiv \sub{p_2}{U_2}
  \\
  \quad \to \subcases ~ p_1 ~ p_2 ~ U_1 ~ U_2
\end{array}
\]
This function demonstrates that when two substitutions are equal, we can always identify a corresponding type from one of the cases defined above. The proof proceeds by pattern-matching on paths $p_1$ and $p_2$. For simple base cases, the proof directly constructs the appropriate case value. For recursive cases, we use Agda's $\mathsf{with}$ to perform intermediate computations and use the intermediate results to construct the desired term in the corresponding type.

\section{Nonassociative Lambek calculus}\label{sec:calculus}
In this section, we present the sequent calculus for the nonassociative Lambek calculus and its cut-elimination property.

A sequent takes the form $T \vd C$, where $T$ is a tree and $C$ is a single formula.
The derivation rules of the calculus are given below:
\begin{equation}\label{eq:seqcalc}
\begin{array}{c}
  \infer[\ax]{X \vd X}{}
  \quad
  \infer[\Rr]{T \vd A \Rarr B}{A , T \vd B} 
  \quad
  \infer[\Lr]{T \vd A \Rarr B}{T , A \vd B \Larr A} 
  \\[7pt]
  \infer[\Rl]{T[U , A \Rarr B] \vd C}{
    U \vd A
    &
    T[B] \vd C
  }
  \quad
  \infer[\Ll]{T[U , B \Larr A] \vd C}{
    U \vd A
    &
    T[B] \vd C
  }
  \\[7pt]
  \infer[\tr]{T , U \vd A \ot B}{
    T \vd A
    &
    U \vd B
  }
  \quad
  \infer[\tl]{T[A \ot B] \vd C}{
    T[A , B] \vd C
  }
\end{array}
\end{equation}

In our Agda implementation, derivations are encoded as an inductive datatype where each rule in (\ref{eq:seqcalc}) corresponds to exactly one constructor. Here we illustrate two examples, $\ax$ and $\Rl$:
\[
\begin{array}{l}
\ax : \forall \{X\} 
\\
\quad \to \eta ~ (\at ~ X) \vd \at ~ X
\end{array}
\]

This constructor specifies that for any atomic formula $X : \At$, there exists a term of type $\eta ~ (\at ~ X) \vd \at ~ X$, directly corresponding to the $\ax$ rule in (\ref{eq:seqcalc}). Since the antecedent is a tree without holes, this construction avoids introducing inconsistent derivations.

The $\Rl$ rule is implemented as:
\[
\begin{array}{rcl}
\Rl & : & \forall ~ \{T ~ U ~ A ~ B ~ C \} \\
&\to& (p : \pathT{T}) \\
&\to&(f : U \vd A) ~ (g : \sub{p}{(\eta ~ B)} \vd C) \\
&\to& \sub{p}{(U\cdast ~ \eta ~ (A \Rarr B))} \vd C
\end{array}
\]

Thus, given terms $f : U \vd A$ and $g : \sub{p}{(\eta ~ B)} \vd C$, the expression $\Rl ~f ~g$ produces a term of type $\sub{p}{(U\cdast ~ \eta ~ (A \Rarr B))} \vd C$.

The admissible cut rule for this calculus takes the form:
\[
\begin{array}{c}
  \infer[\cut]{T[U] \vd C}{
    U \vd D
    &
    T[D] \vd C
  }
\end{array}
\]
In the types of the function below, and further in this paper, we sometimes omit some of the implicit arguments to shorten the types and improve readability.

The $\cut$ rule is implemented as:
\[
\begin{array}{l}
  \cut : (f : U \vd D) ~(g : W \vd C) 
  \\
  \quad \to (eq : W \equiv \sub{p}{(\eta ~ D)})
  \\
  \quad \to \sub{p}{U} \vd C
\end{array}
\]

We introduce an auxiliary term $eq$ to enable Agda to pattern-match on $g$.
The proof proceeds by induction on $g$. Similar to pen-and-paper proofs, in certain cases we need to determine the relative positions of the cut formula and the principal formula. For instance, when $g = \tl ~ g'$, the derivation appears as:
\[
\begin{array}{c}
  \infer[\cut]{T[U] \vd C}{
    \deduce{U \vd D}{f}
    &
    \infer[\tl]{T'[A \ot B] \vd C}{
      \deduce{T'[A , B] \vd C}{g'}
    }
  }
\end{array}
\] 

Here, the tree $T'[A\ot B]$ must have another representation as $T''[D]$ for some tree $T''$. In pen-and-paper proofs, we need to determine whether $D$ exactly matches the principal formula $A \ot B$.

In our Agda implementation, we use the $\mf{with}$ construct and the $\subeq$ function to perform intermediate computation and identify the different possible cases.
There are three subcases: $\oneeqtwo$ (where $D$ is exactly the principal formula $A \ot B$), $\oneLtwoR$ (where $D$ and $A \ot B$ are in different subtrees with $A \ot B$ on the left and $D$ on the right), and $\oneRtwoL$ (the dual case).
For the latter two cases, we permute the $\tl$ rule downward and continue recursively.
For the exact match case, we introduce an auxiliary function:
\[
\begin{array}{l}
  \cut\tl : (f : U \vd A \ot B) ~(g : \sub{p}{(\eta ~ A \cdast \eta ~ B)} \vd C) 
  \\
  \quad \to \sub{p}{U} \vd C
\end{array}
\]

This function is proven by pattern-matching on $f$, mirroring the pen-and-paper approach where we further induct on the other derivation when the cut formula and principal formula coincide.
We employ this auxiliary function to avoid termination checking failures in subsequent proofs involving properties of $\cut$.
The same technique of using auxiliary functions is also applied to the rules $\Rl$ and $\Ll$.
\begin{remark}
  An alternative approach would be to prove $\cut$ by first pattern-matching on $f$ and distinguishing cases where $f$ concludes with a right rule. However, this approach results in a longer proof that requires more applications of the $\mf{with}$ construct and the $\subeq$ function, which significantly increases type-checking time of the subsequent proofs of properties related to $\cut$. 
\end{remark}

\section{Equivalence of derivations}
Sets of derivations of \NL\ are quotiented by a congruence relation $\circeq$, generated by the pairs of derivations.
There are twenty-seven cases of permutative conversions, therefore we only present three examples in Figure \ref{fig:example:perm:conversion}, each of which represents left rules permute with right rules, permutation of sequential application of left rules, and permutation of parallel application of left rules.
\begin{figure}[t]
  \[
  \scalebox{0.91}{$
  \begin{array}{rcl}
    \proofbox{
      \infer[\tl]{T[A \ot B] \vd A' \Rarr B'}{
        \infer[\Rr]{T[A , B] \vd A' \Rarr B'}{
          \deduce{A' , T[A , B] \vd B'}{f}
        }
      }
    }
    &\circeq&
    \proofbox{
      \infer[\Rr]{T[A \ot B] \vd A' \Rarr B'}{
        \infer[\tl]{A' , T[A \ot B] \vd B'}{
          \deduce{A' , T[A , B] \vd B'}{f}
        }
      }
    }
    \\
    \proofbox{
      \infer[\tl]{U[T[A' \ot B'], A \Rarr B] \vd C}{
        \infer[\Rl]{U[T[A' , B'], A \Rarr B] \vd C}{
          \deduce{T[A' , B'] \vd A}{f}
          &
          \deduce{U[B] \vd C}{g}
        }
      }
    }
    &\circeq&
    \proofbox{
      \infer[\Rl]{U[T[A' \ot B'], A \Rarr B] \vd C}{
        \infer[\tl]{T[A' \ot B'] \vd A}{
          \deduce{T[A' , B'] \vd A}{f}
        }
        &
        \deduce{U[B] \vd C}{g}
      }
    }
    \\
    \proofbox{
      \infer[\Rl]{T[W_1[U , A \Rarr B] , W_2[V , A' \Rarr B']] \vd C}{
        \deduce{U \vd A}{f}
        &
        \infer[\Rl]{T[W_1[B] , W_2[V , A' \Rarr B']] \vd C}{
          \deduce{V \vd A'}{f'}
          &
          \deduce{T[W_1[B] , W_2[B']] \vd C}{g}
        }
      }
    }
    &\circeq&
    \proofbox{
      \infer[\Rl]{T[W_1[U , A \Rarr B] , W_2[V , A' \Rarr B']] \vd C}{
        \deduce{V \vd A'}{f'}
        &
        \infer[\Rl]{T[W_1[U , A \Rarr B] , W_2[B']] \vd C}{
          \deduce{U \vd A'}{f}
          &
          \deduce{T[W_1[B] , W_2[B']] \vd C}{h}
        }
      }
    }
  \end{array}
  $}
  \]
  \caption{Examples of permutative conversions}
  \label{fig:example:perm:conversion}
\end{figure}

More equivalences and equations on derivations hold in \NL\ due to the cut-elimination procedures defined in Section \ref{sec:calculus}.
The first is the set of equivalences in Figure \ref{fig:cut:left:rules}, which shows that $\cut$ and left rules are permutatable up to $\circeq$. Notice that the cut-elimination procedure is defined by first induction on the right premise while here the permutativity lies for the left premises.
The second is the set of equations in Figure \ref{fig:cut:properties}, which shows that the rule $\ax$ is the unit of $\cut$, sequential composition of $\cut$ is associative, and parallel composition of $\cut$ is commutative.
Notice that pairs of derivations in these equations are \emph{strictly} equal, not merely $\circeq$-related.
  \begin{figure}[t]
  \[
  \scalebox{0.9}{$
  \begin{array}{rcl}
    \proofbox{
        \infer[\cut]{T[U[V , A \Rarr B]] \vd C}{
      \infer[\Rl]{U[V , A \Rarr B] \vd D}{
        \deduce{V \vd A}{f}
        &
        \deduce{U[B] \vd D}{h}
      }
      &
      \deduce{T[D] \vd C}{g}
    }
    }
    &\circeq&
    \proofbox{
      \infer[\Rl]{T[U[V , A \Rarr B]] \vd C}{
        \deduce{V \vd A}{f}
        &
        \infer[\cut]{T[U[B]] \vd C}{
          \deduce{U[B] \vd D}{h}
          &
          \deduce{T[D] \vd C}{g}
        }
      }
    }
    \\[5pt]
    \proofbox{
        \infer[\cut]{T[U[B \Larr A , V]] \vd C}{
      \infer[\Ll]{U[B \Larr A , V] \vd D}{
        \deduce{V \vd A}{f}
        &
        \deduce{U[B] \vd D}{h}
      }
      &
      \deduce{T[D] \vd C}{g}
    }
    }
    &\circeq&
    \proofbox{
      \infer[\Ll]{T[U[B \Larr A , V]] \vd C}{
        \deduce{V \vd A}{f}
        &
        \infer[\cut]{T[U[B]] \vd C}{
          \deduce{U[B] \vd D}{h}
          &
          \deduce{T[D] \vd C}{g}
        }
      }
    }
    \\[5pt]
    \proofbox{
      \infer[\cut]{T[U[A \ot B]] \vd C}{
        \infer[\tl]{U[A \ot B] \vd D}{
          \deduce{U[A , B] \vd D}{h}
        }
        &
        \deduce{T[D] \vd C}{g}
      }
    }
    &\circeq&
    \proofbox{
      \infer[\tl]{T[U[A \ot B]] \vd C}{
        \infer[\cut]{T[U[A , B]] \vd D}{
          \deduce{U[A , B] \vd D}{h}
          &
          \deduce{T[D] \vd C}{g}
        }
      }
    }
  \end{array}
  $}
  \]
    \caption{Permutation of $\cut$ and left rules}
    \label{fig:cut:left:rules}
  \end{figure}
% cut⇒L≗ : ∀ {T U V W A B C D} (p : Path T) (q : Path U)
%   → {f : V ⊢ A}
%   → (h : sub q (η B) ⊢ D)
%   → (g : W ⊢ C)
%   → (eq : W ≡ sub p (η D))
%   → cut p (⇒L q f h) (subst-cxt eq g) refl ≗ ⇒L (p ++ q) f (cut p h (subst-cxt eq g) refl)

% cut⇐L≗ : ∀ {T U V W A B C D} (p : Path T) (q : Path U)
%   → {f : V ⊢ A}
%   → (h : sub q (η B) ⊢ D)
%   → (g : W ⊢ C)
%   → (eq : W ≡ sub p (η D))
%   → cut p (⇐L q f h) (subst-cxt eq g) refl ≗ ⇐L (p ++ q) f (cut p h (subst-cxt eq g) refl)

% cut⊗L≗ : ∀ {T U W A B C D} (p : Path T) (q : Path U)
%   → (h : sub q (η A ⊛ η B) ⊢ D)
%   → (g : W ⊢ C)
%   → (eq : W ≡ sub p (η D))
%   → cut p (⊗L q h) (subst-cxt eq g) refl ≗ ⊗L (p ++ q) (cut p h (subst-cxt eq g) refl)
  \begin{figure}[t]
  \[
  \arraycolsep=2pt
  \scalebox{0.88}{$
  \begin{array}{rcl}
  \proofbox{
      \infer[\cut]{T[X] \vd C}{
        \infer[\ax]{X \vd X}{}
        &
        \deduce{T[X] \vd C}{g}
      }
    }
    &=&
    \proofbox{
      \deduce{T[X] \vd C}{g}
    }
\\
    \proofbox{
      \infer[\cut]{T[U[V]] \vd C}{
        \infer[\cut]{U[V] \vd E}{
          \deduce{V \vd D}{f}
          &
          \deduce{U[D] \vd E}{g}
        }
        &
        \deduce{T[E] \vd C}{h}
      }
    }
    &=&  
    \proofbox{
      \infer[\cut]{T[U[V]] \vd C}{
        \deduce{V \vd D}{f}
        &
        \infer[\cut]{T[U[D]] \vd C}{
          \deduce{U[D] \vd E}{g}
          &
          \deduce{T[E] \vd C}{h}
        }
      }
    }
    \\
    \proofbox{
      \infer[\cut]{W[W_1[T], W_2[U]] \vd C}{
        \deduce{T \vd D}{f}
        &
        \infer[\cut]{W[W_1[D], W_2[U]] \vd C}{
          \deduce{U \vd E}{g}
          &
          \deduce{W[W_1[D], W_2[E]] \vd C}{h}
        }
      }
    }
    &=& 
    \proofbox{
      \infer[\cut]{W[W_1[T], W_2[U]] \vd C}{
        \deduce{U \vd E}{g}
        &
        \infer[\cut]{W[W_1[T], W_2[E]] \vd C}{
          \deduce{T \vd D}{f}
          &
          \deduce{W[W_1[D], W_2[E]] \vd C}{h}
        }
      }
    }
  \end{array}
  $}
  \]
  \caption{Unit, associativity, and commutativity of $\cut$}
  \label{fig:cut:properties}
  \end{figure}
% cut-unit : ∀ {T X W C} (p : Path T)
%   → (g : W ⊢ C)
%   → (eq : W ≡ sub p (η (at X))) 
%   → cut p ax (subst-cxt eq g) refl ≡ (subst-cxt eq g)

% cut-vass : ∀ {T U V W₁ W₂ C D E} (p : Path T) (q : Path U)
%   → (f : V ⊢ D) (g : W₁ ⊢ E) (h : W₂ ⊢ C)
%   → (eq₁ : W₁ ≡ sub p (η D)) (eq₂ : W₂ ≡ sub q (η E))
%   → cut q (cut p f g eq₁) h eq₂ ≡ cut (q ++ p) f (cut q g h eq₂) (cong (λ x → sub q x) eq₁)

% cut-hass : ∀ {T U V W₁ W₂ W₃ C D E} (p₁ : Path W₁) (p₂ : Path W₂) (p₃ : Path W₃)
%   → (f : T ⊢ D) (g : U ⊢ E) (h : V ⊢ C)
%   → (eq : V ≡ sub p₁ (sub p₂ (η D) ⊛ sub p₃ (η E)))
%   → cut (p₁ ++ (p₂ ◂ _)) f (cut (p₁ ++ (_ ▸ p₃)) g h eq) refl ≡ cut (p₁ ++ (_ ▸ p₃)) g (cut (p₁ ++ (p₂ ◂ _)) f h eq) refl

The relation $\circeq$ is implemented as an inductive type family indexed over pairs of derivations.
In the Agda definition of $\circeq$, there is: $(i)$ a constructor for each permutative conversion, $(ii)$  constructors for reflexivity, symmetry and transitivity of $\circeq$, and  $(iii)$ constructors evidencing the compatibility of $\circeq$ with the inference rules of \NL.
We present three indicative examples, one for each class of equational generators: the constructor $\tl\Rr$ associated to the first example of permutative conversion in Figure \ref{fig:example:perm:conversion}; the constructor $\mathsf{sym}{\circeq}$ associated to symmetry of $\circeq$; the constructor $\mathsf{cong}\Rl$ associated to the compatibility of $\circeq$ wrt. $\Rl$.
\[
\begin{array}{rcl}
\tl\Rr &:& (p : \pathT{T}) ~ (f : \eta A' \cdast \sub{p}{(\eta A \cdast \eta B)} \vd B') \\
&\to& \tl ~ p ~ (\Rr ~ f) ~ \circeq \Rr ~ (\tl ~ (\bullet \btright p) ~ f)  \\[2pt]
\mathsf{sym}{\circeq} &:& (f ~ g : T \vd C) ~ (eq : f \circeq g) \to g \circeq f \\[2pt]
\mathsf{cong}\Rl &:& (p : \pathT{T}) ~ (f ~ f' : U \vd A) ~ (g ~ g' : \sub{p}{\eta B} \vd C) \\
& \to & (eq : f \circeq f') ~ (eq' : g \circeq g') \to \, \Rl ~ p ~ f ~ g \circeq \Rl ~ p ~ f' ~ g'
\end{array}
\]

\section{Interpolation properties}
In this section, we introduce three aspects of interpolation for nonassociative Lambek calculus: Maehara interpolation, proof-relevant interpolation, and showing that Maehara interpolation is well-defined with respect to equivalence of derivations.

\emph{Maehara interpolation property} (\MIP) \cite{ono:proof:nonclassical:1998} is a property of a cut-free sequent calculus. 
It originates from Maehara's proof for Craig interpolation for sequent calculus for classical logic $\mathtt{LK}$ \cite{maehara1961} and the method has been subsequently used to prove Craig interpolation for various logics.
For the full Lambek calculus (\FL), the Maehara interpolation property is formulated as follows:
\begin{description}
  \item[(\MIP~for \FL)] Given $f : \GG \vdash C$ and a partition $\langle \GG_0, \GG_1, \GG_2 \rangle$ of $\GG$, there exist a formula $D$ and two derivations $g : \GG_1 \vdash D$ and $h : \GG_0, D, \GG_2 \vdash C$ such that $\vars{D} \subseteq \vars{\GG_0} \cap \vars{\GG_0, \GG_1, C}$
\end{description}
 $\vars{A}$ denotes the set of atomic formulae appearing in $A$, which can be extended naturally to lists of formulae, trees, and contexts.

% \FL\ without additive connectives enjoys an stronger form of Maehara interpolation that replaces variable condition with a more refined variable \emph{multiplicity} condition \cite{moot:categorial:2012}.
% This refinement tracks the precise number of occurrences of each atomic formula. Let us denote by $\gs{A}$ the count of occurrences of atomic formula $X$ within formula $A$, and extend this notation to $\gs{\GG}$ representing the occurrence count of $X$ across a sequence of formulae $\GG$.
% Under this perspective, the strengthened version of Maehara interpolation is formulated as follows:
% \begin{description}
%   \item[(\MIP~for \FL~with variable multiplicity condition)] Given $f : \GG \vdash C$ and a partition $\langle \GG_0, \GG_1, \GG_2 \rangle$ of $\GG$, there exist a formula $D$ and two derivations $g : \GG_1 \vdash D$ and $h : \GG_0, D, \GG_2 \vdash C$ such that $\gs{D} \leq \gs{\GG_1}$ and $\gs{D} \leq \gs{\GG_0, \GG_2 , C}$ for all atomic $X$.
% \end{description}
% In his PhD thesis \cite{roorda1991}, Roorda considered a even more stronger variable multiplicity condition, not merely the number of occurrences of atomic formulae but also characterizing the \emph{positive and negative} occurrences of atomic formulae in interpolants. Let $\mathsf{At}^{+}_{X}(A)$ be the multiset of positive occurrences of the atomic formula $X$ in $A$, and let $\mathsf{At}^{-}_{X}(A)$ be the multiset of negative occurrences of the atomic formula $X$ in $A$.
% Roorda's version of Maehara interpolation can be formulated as follows:
% \begin{description}
%   \item[(\MIP~for \FL~with variable polarity condition)] Given $f : \GG \vdash C$ and a partition $\langle \GG_0, \GG_1, \GG_2 \rangle$ of $\GG$, there exist a formula $D$ and two derivations $g : \GG_1 \vdash D$ and $h : \GG_0, D, \GG_2 \vdash C$ such that given $p \in \{ +,-\}$ and any atomic formula $X$, there is an injective function from $\mathsf{At}^{p}_{X}(D)$ to $\mathsf{At}^{p}_{X}(\GG_1)$, and an injective function from $\mathsf{At}^{p}_{X}(D)$ to the multiset union of $\mathsf{At}^{\neg p}_{X}(\GG_0, \GG_2)$ and $\mathsf{At}^{p}_{X}(C)$. Here $\neg p$ denotes the opposite polarity of $p$, i.e. $\neg + = -$ and $\neg - = +$.
% \end{description}

Interpolation proofs for various extensions of nonassociative Lambek calculus often depart from the Maehara approach. These alternative methods typically rely on identifying a closed set of formulae that can appear in derivations and proving that interpolant formulae always belong to this set \cite{buszkowski:2009,buszkowski:2010}. Such interpolation theorems primarily serve to establish further properties of the target logic, such as the finite model property, rather than directly addressing our current focus.

Maehara interpolation property for the nonassociative Lambek calculus (\ref{eq:seqcalc}) is stated as the following:
\begin{theorem}\label{thm:MIP}
  \item[(\MIP~for nonassociative Lambek calculus)] Given a derivation $f : T[U]\vd C$, there exist a formula $D$ and two derivations $g : T[D] \vd C$ and $h : U \vd D$ such that $\vars{D} \subseteq \vars{U} \cap \vars{T[\bullet], C}$.
\end{theorem}
This property, not including the variable multiplicity condition (which is discussed later), is implemented in Agda as the following record type:
\[
\begin{array}{rl}
  \multicolumn{2}{l}{\record \:\:  \MIP ~(T : \Tree) ~ (p : \pathT{T}) ~ (U : \Tree) ~ (C : \Fma) : \Set \:\: \where} \\
  \multicolumn{2}{l}{\quad \mf{constructor} ~ \mf{intrp}} \\
  \multicolumn{2}{l}{\quad \field} \\
  \;\; \quad D &: \Fma \\[0.5pt]
  \;\; \quad g &: \sub{p}{(\eta ~ D)} \vd C \\[0.5pt]
  \;\; \quad h &: U \vd D 
\end{array}
\]
Elements of type $\MIP ~ T ~ p ~ U ~C$ are triples consisting of a formula $D$ and two derivations $g$ and $h$.

The proof of Theorem \ref{thm:MIP} becomes the construction of the following function:
\[
\begin{array}{l}
\mathsf{mip} : \forall ~ \{T\} ~ (p : \pathT{T}) ~ U ~ \{V ~ C\}
\\
\quad \to  (f : V \vd C) ~ (eq : V \equiv \sub{p}{U})
\\
\quad \to \MIP ~ T ~ p ~ U ~C
\end{array}
\]
The construction of the $\mathsf{mip}$ function proceeds by pattern-matching on the derivation $f$, which mirrors the pen-and-paper proofs in \cite{roorda1991,moot:categorial:2012} that use induction on $f$. Similar to our implementation of cut-elimination, we need to determine whether $U$ and the principal formula of the conclusion coincide. This analysis is handled by the $\mathsf{subeq}$ function, which helps us systematically identify and process all possible cases that may arise during the construction.

The variable condition for \MIP\ is implemented as follows.
We first define datatype $\in^{\mf{T}}$ which captures how atomic formulas can appear in a tree.
\[
\begin{array}{lrl}
  \multicolumn{3}{l}{\data \:\: \_\inT\_ ~ (X : \At) : \Tree \to \Set \:\: \where} \\
  \;\; \quad \at& &: \forall ~ \{A\} \to X \in A \to X \inT \eta A \\
  \;\; \quad \mf{left}& &: \forall \{T ~ U\} \to X \inT T \to X \inT T \cdast U \\
  \;\; \quad \mf{right}& &: \forall \{T ~ U\} \to X \inT U \to X \inT T \cdast U
\end{array}
\]
Each constructor represents a way an atomic formula can appear in a tree: $\at$ indicates that if $X \in \vars{A}$, then $X$ also appears in the set of atomic formulae of the \emph{tree} $A$; $\mf{left}$ means that if $X \in \vars{T}$, then it also appears in the set of atomic formulae of the tree $T \cdast U$ for any $U$; and $\mf{right}$ is the dual case.

When a tree $U$ is substituted into a context $T[\bullet]$ (resulting in $T[U]$), variables can come from either the context or the substituted tree. We capture this relationship with two functions:
\[
\begin{array}{l}
{\in}\mf{sub}_1 : \forall ~ \{A ~ T ~ U \} ~ (p : \pathT{T}) \to A \inT T \to A \inT \sub{p}{U}
\\
{\in}\mf{sub}_2 : \forall ~ \{A ~ T ~ U \} ~ (p : \pathT{T}) \to A \inT U \to A \inT \sub{p}{U}
\end{array}
\]
The function ${\in}\mf{sub}_1$ shows how variables from the original context $T$ propagate to the substituted tree, while ${\in}\mf{sub}_2$ shows how variables from the substituted tree $U$ appear in the substituted tree. Both functions are defined by pattern-matching on the path $p : \pathT{T}$, with ${\in}\mf{sub}_1$ additionally pattern-matching on the witness that $A$ appears in $T$.

With these foundations, we define the variable condition as a record type:
\[
\begin{array}{rl}
  \multicolumn{2}{l}{\record \:\:  \mf{VarCond} ~ (T : \Tree) ~ (p : \pathT{T}) ~ (U : \Tree) ~ (C ~ D : \Fma) : \Set \:\: \where} \\
  \multicolumn{2}{l}{\quad \mf{constructor} ~ \mf{var}} \\
  \multicolumn{2}{l}{\quad \field} \\
  \;\; \quad \mf{varg} &: \forall \{X\} \to X \in D \to X \inT T \uplus X \in C \\[0.5pt]
  \;\; \quad \mf{varh} &: \forall \{X\} \to X \in D \to X \inT U
\end{array}
\]
$\uplus$ denotes disjoint union in Agda.
This record captures the essential variable condition for interpolation: $\mf{varg}$ states that any variable appearing in the interpolant formula $D$ must come from either the context $T[\bullet]$ or the formula $C$, while $\mf{varh}$ states that all variables in the interpolant $D$ must also appear in the tree $U$.

We verify that the interpolant produced by our algorithm satisfies this variable condition through the function:
\[
\begin{array}{l}
\mf{varcond} : \forall ~ \{T\} ~ (p : \pathT{T}) ~ U ~ \{V ~ C\}
\\
\quad \to  (f : V \vd C) ~ (eq : V \equiv \sub{p}{U})
\\
\quad \to \mf{VarCond} ~ T ~ p ~ U ~ C ~ (D ~ (\mf{mip'} ~ p ~ U ~ f ~ eq))
\end{array}
\]
This function constructs a proof of the variable condition for the interpolant generated by $\mf{mip}$. Its implementation follows the structure of $\mf{mip}$, ensuring that the variable condition is preserved through all cases of the interpolation algorithm.

% _∈_ : (X : At) → Fma → Set
% X ∈ at Y = X ≡ Y
% X ∈ (A ⇒ B) = X ∈ A ⊎ X ∈ B
% X ∈ (B ⇐ A) = X ∈ A ⊎ X ∈ B
% X ∈ (A ⊗ B) = X ∈ A ⊎ X ∈ B
  
% data _∈ᵀ_ (X : At) : Tree → Set where
%   at : ∀ {A} → X ∈ A → X ∈ᵀ η A
%   left : ∀ {T U} → X ∈ᵀ T → X ∈ᵀ T ⊛ U
%   right : ∀ {T U} → X ∈ᵀ U → X ∈ᵀ T ⊛ U

% -- _∈ᵀ_ : At → Tree → Set
% -- X ∈ᵀ T = Σ Fma λ A → (X ∈ A) × (A ∈ᵀ T)

% sub∈ : ∀ {A T U} (p : Path T) → A ∈ᵀ sub p U → A ∈ᵀ T ⊎ A ∈ᵀ U
% sub∈ ∙ (at m) = inj₂ (at m)
% sub∈ ∙ (left m) = inj₂ (left m)
% sub∈ ∙ (right m) = inj₂ (right m)
% sub∈ (p ◂ _) (left m) = elim⊎ (λ n → inj₁ (left n)) inj₂ (sub∈ p m)
% sub∈ (p ◂ _) (right m) = inj₁ (right m)
% sub∈ (T ▸ p) (left m) = inj₁ (left m)
% sub∈ (T ▸ p) (right m) = elim⊎ (λ x → inj₁ (right x)) inj₂ (sub∈ p m)

% ∈sub₁ : ∀ {A T U} (p : Path T) → A ∈ᵀ T → A ∈ᵀ sub p U
% ∈sub₁ (p ◂ _) (left m) = left (∈sub₁ p m)
% ∈sub₁ (p ◂ _) (right m) = right m
% ∈sub₁ (_ ▸ p) (left m) = left m
% ∈sub₁ (_ ▸ p) (right m) = right (∈sub₁ p m)

% ∈sub₂ : ∀ {A T U} (p : Path T) → A ∈ᵀ U → A ∈ᵀ sub p U
% ∈sub₂ ∙ m = m
% ∈sub₂ (p ◂ _) m = left (∈sub₂ p m)
% ∈sub₂ (_ ▸ p) m = right (∈sub₂ p m)

% record VarCond (T : Tree) (p : Path T) (U : Tree) (C D : Fma) : Set where
%   constructor var
%   field
%     varg : ∀ {X} → X ∈ D → X ∈ᵀ T ⊎ X ∈ C
%     varh : ∀ {X} → X ∈ D → X ∈ᵀ U

% varcond : ∀ {T} (p : Path T) U {V C}
%   → (f : V ⊢ C) (eq : V ≡ sub p U) 
%   → VarCond T p U C (D (mip' p U f eq))

We have established a procedure $\mathsf{mip}$ for effectively splitting a derivation $f : T[U] \vdash C$ into two derivations $g : U \vdash D$ and $h : T[D] \vdash C$, with $D$ being ``minimal'' in the sense of satisfying appropriate variable conditions. A natural question arises: what happens when we compose derivations $g$ and $h$ using the admissible $\mathsf{cut}$ rule? Intuition suggests that we should recover the original derivation $f$, at least modulo equivalence of derivations $\circeq$.
Similar questions have been considered by {\v{C}}ubri{\'c} \cite{Cubric1994} in the setting of intuitionistic propositional logic and by Saurin \cite{Saurin2024} for (extensions) of classical linear logic. They call \emph{proof-relevant interpolation} the study of interpolation procedures in relationship to cut rules and equivalence of proofs, like our $\circeq$. In particular, {\v{C}}ubri{\'c} and Saurin show that interpolation procedures are in a way ``right inverses'' of cut rules. Here we show the same for nonassociative Lambek calculus: the $\mathsf{mip}$ procedure is a right inverse of the $\mathsf{cut}$ rule.

\begin{theorem}\label{thm:cutIntrp}
Let $g : T[D] \vd C$ and $h : U \vd D$ be the derivations obtained by applying the \MIP~procedure on a derivation $f: T[U] \vd C$. Then $\mf{cut}(g, h) \circeq f$.
\end{theorem}
Given a derivation $f : V \vdash C$ and an equality proof $eq : V \equiv \text{sub} \ p \ U$, we first apply the Maehara interpolation procedure $\mathsf{mip}$ on them, obtaining an interpolant formula $D$ and two derivations $\mathsf{MIP}.g \ (\mathsf{mip} \ f \ eq)$ and $\mathsf{MIP}.h \ (\mathsf{mip} \ f \ eq)$ (the $g$ and $h$ in the statement of $\mathsf{MIP}$ in Theorem 1). Then we apply the $\mathsf{cut}$ rule on these, resulting in a derivation which is $\circeq$-related to the original $f$. In Agda, proving proof-relevant interpolation translates to the construction of a term $\mathsf{cut\text{-}intrp}$ with the following type:
\[
\begin{array}{l}
\mathsf{cut\text{-}intrp} : \forall ~ \{T\} ~ (p : \pathT{T}) ~ U ~ \{V ~ C\}
\\
\quad \to (f : V \vdash C) ~ (eq : V \equiv \text{sub} ~ p ~ U) 
\\
\quad \to \mathsf{cut} ~ (\mathsf{MIP}.h ~ (\mathsf{mip} ~ f ~ eq)) ~ (\mathsf{MIP}.g ~ (\mathsf{mip} ~ f ~ eq)) ~ \mf{refl} 
\\
\quad \quad \circeq 
\\
\quad \quad \mathsf{subst\text{-}cxt} ~ eq ~ f
\end{array}
\]
% cut-intrp : ∀ {T} (p : Path T) U {V C}
%   → (f : V ⊢ C) (eq : V ≡ sub p U) 
%   →  cut p (MIP.h (mip p U f eq)) (MIP.g (mip p U f eq)) refl ≗ subst-cxt eq f
Notice that $f$ is a term of type $V \vdash C$, while the derivation on the left-hand-side of $\circeq$ has type $\text{sub} \ p \ U \vdash C$. In order to state that the two derivations are $\circeq$-related we need to substitute the context in the type of $f$ using the equality proof $eq : V \equiv \text{sub} \ p \ U$.
The definition of $\mathsf{cut\text{-}intrp}$ proceeds by pattern-matching on the argument $f$.

We define an interpolant triple $(D, g, h)$ as the formulae $D$ together with two derivations $g$ and $h$ that result from applying Maehara interpolation to a derivation $f: T[U] \vdash C$. Two interpolant triples $(D, g, h)$ and $(D', g', h')$ are considered equivalent when $D = D'$, $g \circeq g'$, and $h \circeq h'$.

This equivalence relation between interpolant triples leads to a theorem connecting interpolation and equivalence of derivations: when two derivations $f, f': T[U] \vdash C$ are equivalent (i.e. $f \circeq f'$), their corresponding interpolant triples must also be equivalent. Formally:
\begin{theorem}
  Given two derivations $f, f' : T[U] \vdash C$, if $f \circeq f'$, then the interpolant triples of $f$ and $f'$ are equivalent.
\end{theorem}
In Agda, the property is implemented as the following record type:
\[
\begin{array}{rl}
  \multicolumn{2}{l}{\record \:\:  \MIPeq : \Set \:\: \where} \\
  \multicolumn{2}{l}{\quad \mf{constructor} ~ \mf{intrp}{\circeq}} \\
  \multicolumn{2}{l}{\quad \field} \\
  \;\; \quad eqD &: D \equiv D' \\[0.5pt]
  \;\; \quad eqg &: \mf{subst\text{-}cxt} ~ (\mf{cong} ~ (\Gl x \to \sub{p}{(\eta x)}) ~ eqD) ~ g \circeq g' \\[0.5pt]
  \;\; \quad eqh &: \mf{subst\text{-}succ} ~ eqD ~ h \circeq h'
\end{array}
\]
Elements of type $\MIP ~ T ~ p ~ U ~C$ are triples consisting of a formula $D$ and two derivations $g$ and $h$.

The proof of Theorem \ref{thm:MIP} becomes the construction of the following function:
\[
\begin{array}{l}
\mathsf{mip{\circeq}} : \forall ~ \{T\} ~ (p : \pathT{T}) ~ U ~ \{V ~ C\} ~ \{f ~ f' : V \vd C\}
\\[0.8pt]
\quad \to  (eq_1 : V \equiv \sub{p}{U})
\\[0.8pt]
\quad \to (eq_2 : f \circeq f')
\\[0.8pt]
\quad \to \MIPeq ~ T ~ p ~ U ~ C ~ (\mf{mip} ~ p ~ U ~ f ~ eq_1) ~ (\mf{mip} ~ p ~ U ~ f' ~ eq_1)
\end{array}
\]
We construct $\mathsf{mip{\circeq}}$ by pattern-matching on $eq_2$, the equivalence of derivations. 
The main challenge is handling the thirty-six cases that arise from the equations making $\circeq$ an equivalence relation and a congruence relations wrt. to the logical rules and permutative conversion equations, further complicated by determining the relative positions between interpolant formulae and principal formulae. This proof is difficult to complete accurately on pen-and-paper and prone to errors. Agda is essential here, helping us examine all possible cases systematically and providing formal verification of the completed proof.

\section{Concluding Remarks}

In this paper, we presented an Agda formalization of \NL, focusing on three major proof-theoretic properties: cut-elimination, Maehara interpolation, and proof-relevant interpolation. Our key contribution is the formal characterization of case distinctions for binary trees, which serves as a foundation for rigorously proving properties of \NL.

While cut-elimination and the standard Maehara interpolation property for \NL\ have been established in previous research, our work still offers several contributions. First, our Agda formalization of the case distinction type for trees provides a rigorous foundation that can be reused in further research such as uniform interpolation \cite{alizadeh:2014}. Second, our proof-relevant interpolation results, demonstrating that the Maehara interpolation procedure is well-defined with respect to equivalence of derivations, is new theoretical result of \NL.

Our formalization implements only the basic variable condition for interpolation. The condition can be strengthened to include variable multiplicity \cite{moot:categorial:2012}, or even variable polarity conditions \cite{roorda1991}. However the implementations in Agda are not yet clear, so we leave them as future work.

The well-definedness result for the Maehara interpolation procedure with respect to equivalence of derivations is attainable for nonassociative Lambek calculus because the logic does not contain complicated equivalences of formulae, making the interpolation triple straightforward to define. We anticipate that extending this approach to the associative or semi-associative Lambek calculus \cite{VW2025} would be more challenging, as those systems involve more complex equational theories on formulae.

From a practical perspective, it is worth mentioning that the formalization of these proofs is computationally intensive. The type-checking time for certain files is substantial; the well-definedness proofs (\href{https://github.com/cswphilo/nonassociative-Lambek/blob/main/code/IntrpWellDefined.agda}{$\mf{IntrpWellDefined.agda}$}) require approximately ten minutes to type-check, while the associativity of $\cut$ (\href{https://github.com/cswphilo/nonassociative-Lambek/blob/main/code/CutAssociativities.agda}{$\mf{CutAssociativities.agda}$}) and the congruence of $\cut$ wrt. $\circeq$ (\href{https://github.com/cswphilo/nonassociative-Lambek/blob/main/code/CutCongruence.agda}{$\mf{CutCongruence.agda}$}) require around two hours and six hours on a MacBook Pro with 2 GHz Intel Core i5.

\begin{credits}
\subsubsection{\ackname} This work was supported by the Estonian Research Council grant PSG749. 

% \subsubsection{\discintname}
% It is now necessary to declare any competing interests or to specifically
% state that the authors have no competing interests. Please place the
% statement with a bold run-in heading in small font size beneath the
% (optional) acknowledgments\footnote{If EquinOCS, our proceedings submission
% system, is used, then the disclaimer can be provided directly in the system.},
% for example: The authors have no competing interests to declare that are
% relevant to the content of this article. Or: Author A has received research
% grants from Company W. Author B has received a speaker honorarium from
% Company X and owns stock in Company Y. Author C is a member of committee Z.
\end{credits}
%
% ---- Bibliography ----
%
% BibTeX users should specify bibliography style 'splncs04'.
% References will then be sorted and formatted in the correct style.
%
\bibliographystyle{splncs04}
\bibliography{tableaux}
%
\end{document}
