% This is samplepaper.tex, a sample chapter demonstrating the
% LLNCS macro package for Springer Computer Science proceedings;
% Version 2.21 of 2022/01/12
%
\documentclass[runningheads]{llncs}
%
\usepackage[T1]{fontenc}
% T1 fonts will be used to generate the final print and online PDFs,
% so please use T1 fonts in your manuscript whenever possible.
% Other font encondings may result in incorrect characters.
%
\usepackage{graphicx}
% Used for displaying a sample figure. If possible, figure files should
% be included in EPS format.
%
% If you use the hyperref package, please uncomment the following two lines
% to display URLs in blue roman font according to Springer's eBook style:
%\usepackage{color}
%\renewcommand\UrlFont{\color{blue}\rmfamily}
%\urlstyle{rm}
%
\usepackage{amsmath,amssymb,amsfonts}%
%% macros for math symbols
\newcommand{\ot}{\otimes}
\newcommand{\cdast}{\circledast}
\newcommand{\Larr}{\Leftarrow}
\newcommand{\Rarr}{\Rightarrow}
\newcommand{\btleft}{\blacktriangleleft}
\newcommand{\btright}{\blacktriangleright}

%% commands for Agda stuff
\newcommand{\At}{\mathsf{At}}
\newcommand{\at}{\mathsf{at}}
\newcommand{\Fma}{\mathsf{Fma}}
\newcommand{\data}{\mathsf{data}}
\newcommand{\Tree}{\mathsf{Tree}}
\newcommand{\Path}{\mathsf{Path}}
\newcommand{\path}[1]{\mathsf{Path} ~ #1}
\newcommand{\Sub}{\mathsf{sub}}
\newcommand{\sub}[2]{\mathsf{sub} ~ #1 ~ #2}
\newcommand{\where}{\mathsf{where}}
\newcommand{\Set}{\mathsf{Set}}
\newcommand{\record}{\mathsf{record}}
\newcommand{\field}{\mathsf{field}}

\newcommand{\niccolo}[1]{\textcolor{red}{NV: #1}}
\newcommand{\cheng}[1]{\textcolor{blue}{CSW: #1}}

\begin{document}
%
\title{Contribution Title}
%
%\titlerunning{Abbreviated paper title}
% If the paper title is too long for the running head, you can set
% an abbreviated paper title here
%
\author{Niccol{\'o} Veltri\orcidID{0000-0002-7230-3436} \and
Cheng-Syuan Wan \orcidID{0000-0003-2053-1688}}
%
\authorrunning{N. Velri and C.-S. Wan}
% First names are abbreviated in the running head.
% If there are more than two authors, 'et al.' is used.
%
\institute{Tallinn University of Technology, Tallinn, Estonia
\\
\email{\{niccolo,cswan\}@cs.ioc.ee}}
%
\maketitle              % typeset the header of the contribution
%
\begin{abstract}
The abstract should briefly summarize the contents of the paper in
150--250 words.

\keywords{Nonassociative Lambek calculus \and Agda \and Cut-elimination \and Craig interpolation}
\end{abstract}
%
%
%

\begin{enumerate}
  \item Introduction to Agda formalization.
  \begin{itemize}
    \item Formulae, trees, and substitution.
    \item How to figure out the accurate relationship between two equal trees with substitution.
  \end{itemize}
  \item The target calculus, nonassociative Lambek calculus.
  \begin{itemize}
    \item Rules (including cut), and then cut-elimination.
    \item List some of permutative conversions and leave the full list in the appendix.
    \item Statement of equations and properties of cut and the equivalence relation.
  \end{itemize}
  \item Proof-relevant interpolation.
  \begin{itemize}
    \item Statement of Maehara interpolation (from Roorda or someone else) and its translation in Agda. Maybe a proof sketch.
    \item Proof-relevant interpolation (cut-intrp). Statement and sketch of proof.
    \item Well-definedness of interpolation. Statement and sketch of proof.
  \end{itemize}
  \item Discussion.
  \begin{itemize}
    \item Type-checking takes a long time.
    \item This is the first step of the big project on proof-relevant interpolation. In this calculus, the equivalence relations on interpolation triple is simple to define. In other words, the interpolant formulae are always identical for any two equivalent derivations while it is not the case in the associative Lambek calculus (give an example).
  \end{itemize}
\end{enumerate}

\section{Introduction}

\section{Trees and Substitution}
In this section, we present the Agda formalization of trees, paths in trees and substitutions.
% The formalization of trees and substitution are inspired by the ordinary formalization \cite{moot:categorial:2012}.
\noindent\textbf{Formulae, trees and paths.} In the formalization we fix a type $\At$ of atomic formulae.
The type $\Fma$ of formulae is the following inductive type:
\[
\begin{array}{rl}
  \multicolumn{2}{l}{\data \:\:  \Fma : \Set \:\: \where} \\
  \;\; \at &: \At \to \Fma \\
  \;\; \_{\Larr}\_ &: \Fma \to \Fma \to \Fma \\ 
  \;\; \_{\Rarr}\_ &: \Fma \to \Fma \to \Fma \\ 
  \;\; \_{\ot}\_ &: \Fma \to \Fma \to \Fma \\
\end{array}
\]
Underscores are used to represent infix operators, e.g. $A \ot B$ is a formula for all $A,B : \Fma$.

The type $\Tree$ of trees is the following inductive type:
\[
\begin{array}{rl}
  \multicolumn{2}{l}{\data \:\:  \Tree : \Set \:\: \where} \\
  \;\; \cdot &: \Tree \\
  \;\; \eta &: \Fma \to \Tree \\
  \;\; \_{\cdast}\_ &: \Tree \to \Tree \to \Tree \\
\end{array}
\]
Different from the definitions of trees used for nonassociative Lambek calculus on pen and paper in \cite{moot:categorial:2012}, here holes are also considered as trees. This would not lead to inconsistent derivations because the axiom rules are defined with non-empty antecedents, see Section \ref{sec:calculus}.

The type $\Path$ of paths of trees is the following inductive type:
\[
\begin{array}{rl}
  \multicolumn{2}{l}{\data \:\:  \Path : \Tree \to \Set \:\: \where} \\
  \;\; \cdot &: \path{\cdot} \\
  \;\; \_\btleft\_ &: \forall ~ \{T\} ~ (p : \path{T}) ~ U \to \path{(T \cdast U)} \\
  \;\; \_\btright\_ &: \forall ~ T ~ \{U\} ~ (p : \path{U}) \to \path{(T \cdast U)}
\end{array}
\]
Curly brackets are used in Agda to denote implicit arguments.

Substitution function aims to substitute a tree for a hole in a tree. It is constructed by pattern-matching (i.e. structural recursion) on the path to a specific hole of a tree.
\[
\begin{array}{ll}
  \multicolumn{2}{l}{\Sub : \forall ~ \{T\} \to \path{T} \to \Tree \to \Tree}
  \\[2pt]
  \sub{\cdot}{U} &= U
  \\
  \sub{p \btleft V}{U} &= \sub{p}{U} \cdast V
  \\
  \sub{V \btright p}{U} &= V \cdast \sub{p}{U} 
\end{array}
\]


\section{Nonassociative Lambek calculus}\label{sec:calculus}

\section{Properties of Cut}

\section{Interpolation}

\section{Concluding Remarks}

\begin{credits}
\subsubsection{\ackname} This work was supported by the Estonian Research Council grant PSG749. 

% \subsubsection{\discintname}
% It is now necessary to declare any competing interests or to specifically
% state that the authors have no competing interests. Please place the
% statement with a bold run-in heading in small font size beneath the
% (optional) acknowledgments\footnote{If EquinOCS, our proceedings submission
% system, is used, then the disclaimer can be provided directly in the system.},
% for example: The authors have no competing interests to declare that are
% relevant to the content of this article. Or: Author A has received research
% grants from Company W. Author B has received a speaker honorarium from
% Company X and owns stock in Company Y. Author C is a member of committee Z.
\end{credits}
%
% ---- Bibliography ----
%
% BibTeX users should specify bibliography style 'splncs04'.
% References will then be sorted and formatted in the correct style.
%
\bibliographystyle{splncs04}
\bibliography{tableaux}
%
\end{document}
