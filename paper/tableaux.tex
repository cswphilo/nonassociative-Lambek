% This is samplepaper.tex, a sample chapter demonstrating the
% LLNCS macro package for Springer Computer Science proceedings;
% Version 2.21 of 2022/01/12
%
\documentclass[runningheads]{llncs}
%
\usepackage[T1]{fontenc}
% T1 fonts will be used to generate the final print and online PDFs,
% so please use T1 fonts in your manuscript whenever possible.
% Other font encondings may result in incorrect characters.
%
\usepackage{hyperref}
\usepackage{graphicx}
\usepackage{quiver}
\usepackage{proof}
\usepackage{tikz-cd}
% \tikzcdset{scale cd/.style={every label/.append style={scale=#1},
    % cells={nodes={scale=#1}}}}
\tikzcdset{row sep/normal=0.4cm}
\tikzcdset{column sep/normal=0.02cm}
% Used for displaying a sample figure. If possible, figure files should
% be included in EPS format.
%
% If you use the hyperref package, please uncomment the following two lines
% to display URLs in blue roman font according to Springer's eBook style:
%\usepackage{color}
%\renewcommand\UrlFont{\color{blue}\rmfamily}
%\urlstyle{rm}
%
\usepackage{amsmath,amssymb,amsfonts}%

%% macros for typesetting
\newcommand{\udl}[1]{\underline{#1}}
%% macros for math symbols
\newcommand{\ot}{\otimes}
\newcommand{\cdast}{\circledast}
\newcommand{\Larr}{\Leftarrow}
\newcommand{\Rarr}{\Rightarrow}
\newcommand{\btleft}{\blacktriangleleft}
\newcommand{\btright}{\blacktriangleright}
\newcommand{\sls}{\slash}
\newcommand{\bsls}{\backslash}
\newcommand{\mc}[1]{\mathcal{#1}}
\newcommand{\mf}[1]{\mathsf{#1}}


%% macros for derivations
\newcommand{\vd}{\vdash}
\newcommand{\ax}{\mathsf{ax}}
\newcommand{\tl}{\otimes \mathsf{L}}
\newcommand{\tr}{\otimes\mathsf{R}}
\newcommand{\Ll}{{\Larr}\mathsf{L}}
\newcommand{\Lr}{{\Larr}\mathsf{R}}
\newcommand{\Rl}{{\Rarr}\mathsf{L}}
\newcommand{\Rr}{{\Rarr}\mathsf{R}}
\newcommand{\cut}{\mf{cut}}

%% commands for Agda stuff
\newcommand{\At}{\mathsf{At}}
\newcommand{\at}{\mathsf{at}}
\newcommand{\Fma}{\mathsf{Fma}}
\newcommand{\data}{\mathsf{data}}
\newcommand{\Tree}{\mathsf{Tree}}
\newcommand{\Path}{\mathsf{Path}}
\newcommand{\pathT}[1]{\mathsf{Path} ~ #1}
\newcommand{\append}{+\!\!+}
\newcommand{\Sub}{\mathsf{sub}}
\newcommand{\sub}[2]{\mathsf{sub} ~ #1 ~ #2}
\newcommand{\where}{\mathsf{where}}
\newcommand{\Set}{\mathsf{Set}}
\newcommand{\record}{\mathsf{record}}
\newcommand{\field}{\mathsf{field}}
\newcommand{\subst}{\mathsf{subst}}
\newcommand{\Same}{\mathsf{Same}}
\newcommand{\ContainsLeft}{\mathsf{ContainsLeft}}
\newcommand{\ContainsRight}{\mathsf{ContainsRight}}
\newcommand{\LeftRight}{\mathsf{LeftRight}}
\newcommand{\subcases}{\mathsf{SubEqCases}}
\newcommand{\oneeqtwo}{\mathsf{1{\equiv}2}}
\newcommand{\twogtLone}{\mathsf{2{>}L1}}
\newcommand{\twogtRone}{\mathsf{2{>}R1}}
\newcommand{\onegtLtwo}{\mathsf{1{>}L2}}
\newcommand{\onegtRtwo}{\mathsf{1{>}R2}}
\newcommand{\oneLtwoR}{\mathsf{1\sls \bsls 2}}
\newcommand{\oneRtwoL}{\mathsf{2\sls \bsls 1}}
\newcommand{\subeq}{\mathsf{subeq}}


\newcommand{\niccolo}[1]{\textcolor{red}{NV: #1}}
\newcommand{\cheng}[1]{\textcolor{blue}{CSW: #1}}

\begin{document}
%
\title{Contribution Title}
%
%\titlerunning{Abbreviated paper title}
% If the paper title is too long for the running head, you can set
% an abbreviated paper title here
%
\author{Niccol{\'o} Veltri\orcidID{0000-0002-7230-3436} \and
Cheng-Syuan Wan \orcidID{0000-0003-2053-1688}}
%
\authorrunning{N. Velri and C.-S. Wan}
% First names are abbreviated in the running head.
% If there are more than two authors, 'et al.' is used.
%
\institute{Tallinn University of Technology, Tallinn, Estonia
\\
\email{\{niccolo,cswan\}@cs.ioc.ee}}
%
\maketitle              % typeset the header of the contribution
%
\begin{abstract}
The abstract should briefly summarize the contents of the paper in
150--250 words.

\keywords{Nonassociative Lambek calculus \and Agda \and Cut-elimination \and Craig interpolation}
\end{abstract}
%
%
%

\begin{enumerate}
  \item Introduction to Agda formalization.
  \begin{itemize}
    \item Formulae, trees, and substitution.
    \item How to figure out the accurate relationship between two equal trees with substitution.
  \end{itemize}
  \item The target calculus, nonassociative Lambek calculus.
  \begin{itemize}
    \item Rules (including cut), and then cut-elimination.
    \item List some of permutative conversions and leave the full list in the appendix.
    \item Statement of equations and properties of cut and the equivalence relation.
  \end{itemize}
  \item Proof-relevant interpolation.
  \begin{itemize}
    \item Statement of Maehara interpolation (from Roorda or someone else) and its translation in Agda. Maybe a proof sketch.
    \item Proof-relevant interpolation (cut-intrp). Statement and sketch of proof.
    \item Well-definedness of interpolation. Statement and sketch of proof.
  \end{itemize}
  \item Discussion.
  \begin{itemize}
    \item Type-checking takes a long time.
    \item This is the first step of the big project on proof-relevant interpolation. In this calculus, the equivalence relations on interpolation triple is simple to define. In other words, the interpolant formulae are always identical for any two equivalent derivations while it is not the case in the associative Lambek calculus (give an example).
  \end{itemize}
\end{enumerate}

\section{Introduction}

\section{Trees and Substitution}\label{sec:agda:base}
In this section, we present the Agda formalization of trees, paths in trees and substitutions.
% The formalization of trees and substitution are inspired by the ordinary formalization \cite{moot:categorial:2012}.

\noindent\textbf{Formulae, trees and paths.}
In the original paper, formulae are inductively generated by the grammar $A, B ::= X \ | \ A \Larr B \ | \ B \Rarr A \ | \ A \ot B$, where $X$ is drawn from a set $\mathsf{At}$ of atomic formulae.
In our formalization, we postulate a type $\At$ of atomic formulae. We consistently use $X, Y, Z, \ldots$ to represent atomic formulae.
The type $\Fma$ of formulae is defined as the following inductive type:
\[
\begin{array}{rl}
  \multicolumn{2}{l}{\data \:\:  \Fma : \Set \:\: \where} \\
  \;\; \at &: \At \to \Fma \\
  \;\; \_{\Larr}\_ &: \Fma \to \Fma \to \Fma \\ 
  \;\; \_{\Rarr}\_ &: \Fma \to \Fma \to \Fma \\ 
  \;\; \_{\ot}\_ &: \Fma \to \Fma \to \Fma \\
\end{array}
\]
Note that underscores indicate infix operators, so $A \ot B$ represents a formula for any $A,B : \Fma$.

In \cite{moot:categorial:2012}, trees are defined inductively by the grammar $T ::= \Fma \mid (T, T)$.
A context is defined as a tree with a hole, represented recursively as $\mathcal{C} ::= [\cdot] \mid (\mathcal{C}, T) \mid (T, \mathcal{C})$. 

We formalize the type $\Tree$ of trees as the following inductive type:
\[
\begin{array}{rl}
  \multicolumn{2}{l}{\data \:\:  \Tree : \Set \:\: \where} \\
  \;\; \bullet &: \Tree \\
  \;\; \eta &: \Fma \to \Tree \\
  \;\; \_{\cdast}\_ &: \Tree \to \Tree \to \Tree \\
\end{array}
\]

Our definition more closely resembles the definition of contexts in \cite{moot:categorial:2012}. This approach facilitates smoother proofs and allows us to define derivations using the same type. This definition does not introduce inconsistent derivations since the axiom sequents are defined with non-empty antecedents (see Section \ref{sec:calculus} for the formal sequent calculus).


The type $\Path$ of paths of trees is the following inductive type:
\[
\begin{array}{rl}
  \multicolumn{2}{l}{\data \:\:  \Path : \Tree \to \Set \:\: \where} \\
  \;\; \bullet &: \pathT{\bullet} \\
  \;\; \_\btleft\_ &: \forall ~ \{T\} ~ (p : \pathT{T}) ~ U \to \pathT{(T \cdast U)} \\
  \;\; \_\btright\_ &: \forall ~ T ~ \{U\} ~ (p : \pathT{U}) \to \pathT{(T \cdast U)}
\end{array}
\]
Curly brackets are used in Agda to denote implicit arguments.

\noindent\textbf{Substitution and equality of trees.}
The substitution of a tree into a hole is defined recursively:
\begin{displaymath}
  \begin{array}{rcl}
  subst([\cdot], U) &=& U
  \\
  subst((\mc{C},V), U) &=& (subst(\mc{C},U),V)
  \\
  subst((V,\mc{C}), U) &=& (V , subst(\mc{C},U))
  \end{array}
\end{displaymath}
We use $T[\cdot]$ to denote a context and $T[U]$ to abbreviate $subst(T[\cdot], U)$.

In Agda, substitution function is constructed by pattern-matching (i.e. structural recursion) on the path to a specific hole of a tree.
\[
\begin{array}{ll}
  \multicolumn{2}{l}{\Sub : \forall ~ \{T\} \to \pathT{T} \to \Tree \to \Tree}
  \\[2pt]
  \sub{\bullet}{U} &= U
  \\
  \sub{p \btleft V}{U} &= \sub{p}{U} \cdast V
  \\
  \sub{V \btright p}{U} &= V \cdast \sub{p}{U} 
\end{array}
\]

\begin{example}\label{example:tree:and:path}
The following tree is encoded as $(\eta X ~ \cdast ~ \bullet) \cdast ~ \eta Y$ in Agda.
The path to the hole is expressed as $(\eta X \btright \bullet) \btleft \eta Y$ which indicates that starting from the root node, we traverse one step left followed by one step right to reach the hole.
\[
  % https://q.uiver.app/#q=WzAsNSxbMCw0LCJcXGV0YSBYIl0sWzEsMiwiXFxjZGFzdCJdLFsyLDQsIlxcY2RvdCJdLFsyLDAsIlxcY2Rhc3QiXSxbMywyLCJcXGV0YSBZIl0sWzEsMCwiIiwwLHsic3R5bGUiOnsiaGVhZCI6eyJuYW1lIjoibm9uZSJ9fX1dLFsxLDIsIiIsMix7InN0eWxlIjp7ImhlYWQiOnsibmFtZSI6Im5vbmUifX19XSxbMywxLCIiLDIseyJzdHlsZSI6eyJoZWFkIjp7Im5hbWUiOiJub25lIn19fV0sWzMsNCwiIiwwLHsic3R5bGUiOnsiaGVhZCI6eyJuYW1lIjoibm9uZSJ9fX1dXQ==
\begin{tikzcd}
% [sep=tiny]
	&& \cdast \\
	\\
	& \cdast && {\eta Y} \\
	\\
	{\eta X} && \bullet
	\arrow[no head, from=1-3, to=3-2]
	\arrow[no head, from=1-3, to=3-4]
	\arrow[no head, from=3-2, to=5-1]
	\arrow[no head, from=3-2, to=5-3]
\end{tikzcd}
\]
For any tree $U$, the substitution operation that replaces $\bullet$ with $U$ follows the path specified above.
\end{example}
Given two paths $p : \pathT{T}$ and $q : \pathT{U}$, we write $p \append ~ q : \pathT{\sub{p}{U}}$ for the concatenation.
Notice that the concatenation of the two paths is a path in the new tree $\sub{p}{U}$.
\begin{example}\label{example:path:concatenation}
  Recall the tree and path in Example \ref{example:tree:and:path}. We call them $T$ and $p$, respectively.
  Given another tree $U = \eta Z \cdast \bullet$ with the path $q = \eta Z \btright \bullet$, the path $p \append ~ q = (\eta X \btright (\eta Z \btright \bullet)) \btleft \eta Y$, which is a path in the tree $\sub{p}{U}$.
\end{example}
When proving either cut-elimination or interpolation for nonassociative Lambek calculus, a key step is determining different presentations of the same tree. For cut-elimination, this involves finding the relationship between the principal formula of the endsequent and the cut formula. For interpolation, it concerns the relationship with the interpolating tree. 
\begin{example}\label{example:same:tree:diff:sub}
  Considering the tree in Example \ref{example:tree:and:path}, we can represent it in at least two equivalent ways: $\sub{((\eta X \cdast \bullet) \btright \bullet)}{\eta Y}$ or $\sub{(\bullet \btleft \eta Y)}{(\eta X \cdast \bullet)}$.
\end{example}
We formalize this common process by constructing record types that encompass all possible scenarios where the same tree can be presented in two different ways.
Given two paths $p_1 : \pathT{T_1}$ and $p_2 : \pathT{T_2}$ and two trees $U_1$ and $U_2$, if $\sub{p_1}{U_1}$ is equal to $\sub{p_2}{T_2}$, then there are four possibilities:
\\
\udl{$U_1$ is equal to $U_2$.}
In this case, if $\sub{p_1}{U_1}$ is equal to $\sub{p_2}{T_2}$, then $p_1$ and $p_2$ are forced to be equal as well as $T_1$ and $T_2$.
\[
\begin{array}{rl}
  \multicolumn{2}{l}{\record \:\:  \Same ~ (U_1 ~ U_2 : \Tree) : \Set \:\: \where} \\
  \multicolumn{2}{l}{\quad \mathsf{constructor} ~ \mathsf{same}} \\
  \multicolumn{2}{l}{\quad \field} \\
  \;\; \quad eqT &: T_1 \equiv T_2 \\
  \;\; \quad eqU &: U_1 \equiv U_2 \\
  \;\; \quad eqp &: \subst ~ Path ~ eqT ~ p_1 \equiv p_1
\end{array}
\]
Notice that given $x,y : A$, their propositional equality is denoted as $x \equiv y$.
Elements in this type consist of three equalities about original trees, trees used for substitution into holes, and paths, respectively.
\\
\udl{$U_1$ contains $U_2$ in its left subtree.}
In this case, there exists a path in the left subtree of $U_1$ that points to where $U_2$ should be substituted. Put simply, $p_2$ equals $p_1$ extended with the path in the left subtree of $U_1$ without $U_2$ (for brevity, when we say ``left subtree of $U_1$'', we mean the left subtree of $U_1$ minus $U_2$).
Let $W_1$ and $W_2$ be the left and right subtrees of $U_1$, respectively.
Consider trees $T_1$ and $\sub{p_1}{U_1}$ with the following structure (where $V_1$ is an arbitrary tree):

\[
\arraycolsep=1.5cm
\begin{array}{cc}
  % https://q.uiver.app/#q=WzAsMyxbMSwwLCJcXGNkYXN0Il0sWzAsMiwiXFxidWxsZXQiXSxbMiwyLCJWXzEiXSxbMCwxLCIiLDAseyJzdHlsZSI6eyJoZWFkIjp7Im5hbWUiOiJub25lIn19fV0sWzAsMSwicF8xIiwyLHsib2Zmc2V0IjoxLCJjb2xvdXIiOlswLDYwLDYwXSwic3R5bGUiOnsiaGVhZCI6eyJuYW1lIjoibm9uZSJ9fX0sWzAsNjAsNjAsMV1dLFswLDIsIiIsMix7InN0eWxlIjp7ImhlYWQiOnsibmFtZSI6Im5vbmUifX19XV0=
\begin{tikzcd}
	& \cdast \\
	\\
	\bullet && {V_1}
	\arrow[no head, from=1-2, to=3-1]
	\arrow["{p_1}"', shift right, color={rgb,255:red,214;green,92;blue,92}, no head, from=1-2, to=3-1]
	\arrow[no head, from=1-2, to=3-3]
\end{tikzcd}
&
% https://q.uiver.app/#q=WzAsNSxbMiwwLCJcXGNkYXN0Il0sWzEsMiwiXFxjZGFzdCJdLFszLDIsIlZfMSJdLFswLDQsIldfMSJdLFsyLDQsIldfMiJdLFswLDEsIiIsMCx7InN0eWxlIjp7ImhlYWQiOnsibmFtZSI6Im5vbmUifX19XSxbMCwxLCJwXzEiLDIseyJvZmZzZXQiOjEsImNvbG91ciI6WzAsNjAsNjBdLCJzdHlsZSI6eyJoZWFkIjp7Im5hbWUiOiJub25lIn19fSxbMCw2MCw2MCwxXV0sWzAsMiwiIiwyLHsic3R5bGUiOnsiaGVhZCI6eyJuYW1lIjoibm9uZSJ9fX1dLFsxLDMsIiIsMix7InN0eWxlIjp7ImhlYWQiOnsibmFtZSI6Im5vbmUifX19XSxbMSw0LCIiLDIseyJzdHlsZSI6eyJoZWFkIjp7Im5hbWUiOiJub25lIn19fV1d
\begin{tikzcd}
	&& \cdast \\
	\\
	& \cdast && {V_1} \\
	\\
	{W_1} && {W_2}
	\arrow[no head, from=1-3, to=3-2]
	\arrow["{p_1}"', shift right, color={rgb,255:red,214;green,92;blue,92}, no head, from=1-3, to=3-2]
	\arrow[no head, from=1-3, to=3-4]
	\arrow[no head, from=3-2, to=5-1]
	\arrow[no head, from=3-2, to=5-3]
\end{tikzcd}
\\
(T_1)
&
(\sub{p_1}{U_1})
\end{array}
\]
Since $U_1$ contains $U_2$ in its structure, we can diagram $\sub{p_2}{T_2}$ as shown in the left diagram below, where the blue path represents $p_2$. The right diagram illustrates how $p_2$ extends $p_1$. We denote the path in $W_1$ as $q$
\begin{equation}\label{eq:ContainsLeft:p1p2}
  \arraycolsep=1cm
\begin{array}{cc}
% https://q.uiver.app/#q=WzAsNyxbMCw2LCJVXzIiXSxbMSw0LCJXXzEiXSxbMiw2LCJWXzIiXSxbMiwyLCJcXGNkYXN0Il0sWzMsNCwiV18yIl0sWzMsMCwiXFxjZGFzdCJdLFs0LDIsIlZfMSJdLFsxLDAsIiIsMCx7InN0eWxlIjp7ImhlYWQiOnsibmFtZSI6Im5vbmUifX19XSxbMSwyLCIiLDIseyJzdHlsZSI6eyJoZWFkIjp7Im5hbWUiOiJub25lIn19fV0sWzMsMSwiIiwyLHsic3R5bGUiOnsiaGVhZCI6eyJuYW1lIjoibm9uZSJ9fX1dLFszLDQsIiIsMCx7InN0eWxlIjp7ImhlYWQiOnsibmFtZSI6Im5vbmUifX19XSxbNSwzLCIiLDAseyJzdHlsZSI6eyJoZWFkIjp7Im5hbWUiOiJub25lIn19fV0sWzUsNiwiIiwyLHsic3R5bGUiOnsiaGVhZCI6eyJuYW1lIjoibm9uZSJ9fX1dLFs1LDMsIiIsMCx7Im9mZnNldCI6MSwiY29sb3VyIjpbMjQwLDYwLDYwXSwic3R5bGUiOnsiaGVhZCI6eyJuYW1lIjoibm9uZSJ9fX1dLFszLDEsIiIsMCx7Im9mZnNldCI6MSwiY29sb3VyIjpbMjQwLDYwLDYwXSwic3R5bGUiOnsiaGVhZCI6eyJuYW1lIjoibm9uZSJ9fX1dLFsxLDAsIiIsMCx7Im9mZnNldCI6MSwiY29sb3VyIjpbMjQwLDYwLDYwXSwic3R5bGUiOnsiaGVhZCI6eyJuYW1lIjoibm9uZSJ9fX1dXQ==
\begin{tikzcd}
	&&& \cdast \\
	\\
	&& \cdast && {V_1} \\
	\\
	& {W_1} && {W_2} \\
	\\
	{U_2} && {V_2}
	\arrow[no head, from=1-4, to=3-3]
	\arrow[shift right, draw={rgb,255:red,92;green,92;blue,214}, no head, from=1-4, to=3-3]
	\arrow[no head, from=1-4, to=3-5]
	\arrow[no head, from=3-3, to=5-2]
	\arrow[shift right, draw={rgb,255:red,92;green,92;blue,214}, no head, from=3-3, to=5-2]
	\arrow[no head, from=3-3, to=5-4]
	\arrow[no head, from=5-2, to=7-1]
	\arrow[shift right, color={rgb,255:red,92;green,92;blue,214}, no head, from=5-2, to=7-1]
	\arrow[no head, from=5-2, to=7-3]
\end{tikzcd}
&
% https://q.uiver.app/#q=WzAsOCxbMCw2LCJVXzIiXSxbMSw0LCJXXzEiXSxbMiw2LCJWXzIiXSxbMiwyLCJcXGNkYXN0Il0sWzMsNCwiV18yIl0sWzMsMCwiXFxjZGFzdCJdLFs0LDIsIlZfMSJdLFsyLDFdLFsxLDAsIiIsMCx7InN0eWxlIjp7ImhlYWQiOnsibmFtZSI6Im5vbmUifX19XSxbMSwyLCIiLDIseyJzdHlsZSI6eyJoZWFkIjp7Im5hbWUiOiJub25lIn19fV0sWzMsMSwiIiwyLHsic3R5bGUiOnsiaGVhZCI6eyJuYW1lIjoibm9uZSJ9fX1dLFszLDQsIiIsMCx7InN0eWxlIjp7ImhlYWQiOnsibmFtZSI6Im5vbmUifX19XSxbNSwzLCIiLDAseyJzdHlsZSI6eyJoZWFkIjp7Im5hbWUiOiJub25lIn19fV0sWzUsNiwiIiwyLHsic3R5bGUiOnsiaGVhZCI6eyJuYW1lIjoibm9uZSJ9fX1dLFszLDEsIiIsMCx7Im9mZnNldCI6MSwiY29sb3VyIjpbMjQwLDYwLDYwXSwic3R5bGUiOnsiaGVhZCI6eyJuYW1lIjoibm9uZSJ9fX1dLFs1LDMsInBfMSIsMix7Im9mZnNldCI6MSwiY29sb3VyIjpbMCw2MCw2MF0sInN0eWxlIjp7ImhlYWQiOnsibmFtZSI6Im5vbmUifX19LFswLDYwLDYwLDFdXSxbMSwwLCJxIiwyLHsib2Zmc2V0IjoxLCJjb2xvdXIiOlswLDYwLDYwXSwic3R5bGUiOnsiaGVhZCI6eyJuYW1lIjoibm9uZSJ9fX0sWzAsNjAsNjAsMV1dXQ==
\begin{tikzcd}
	&&& \cdast \\
	&& {} \\
	&& \cdast && {V_1} \\
	\\
	& {W_1} && {W_2} \\
	\\
	{U_2} && {V_2}
	\arrow[no head, from=1-4, to=3-3]
	\arrow["{p_1}"', shift right, color={rgb,255:red,214;green,92;blue,92}, no head, from=1-4, to=3-3]
	\arrow[no head, from=1-4, to=3-5]
	\arrow[no head, from=3-3, to=5-2]
	\arrow[shift right, draw={rgb,255:red,92;green,92;blue,214}, no head, from=3-3, to=5-2]
	\arrow[no head, from=3-3, to=5-4]
	\arrow[no head, from=5-2, to=7-1]
	\arrow["q"', shift right, color={rgb,255:red,214;green,92;blue,92}, no head, from=5-2, to=7-1]
	\arrow[no head, from=5-2, to=7-3]
\end{tikzcd}
\end{array}
\end{equation}
We can also see that the trees $T_2$ are $U_1$ the following, respectively:
\begin{equation}\label{eq:ContainsLeft:T2U1}
\arraycolsep=1cm
\begin{array}{cc}
  % https://q.uiver.app/#q=WzAsOCxbMCw2LCJcXGJ1bGxldCJdLFsxLDQsIldfMSJdLFsyLDYsIlZfMiJdLFsyLDIsIlxcY2Rhc3QiXSxbMyw0LCJXXzIiXSxbMywwLCJcXGNkYXN0Il0sWzQsMiwiVl8xIl0sWzIsMV0sWzEsMCwiIiwwLHsic3R5bGUiOnsiaGVhZCI6eyJuYW1lIjoibm9uZSJ9fX1dLFsxLDIsIiIsMix7InN0eWxlIjp7ImhlYWQiOnsibmFtZSI6Im5vbmUifX19XSxbMywxLCIiLDIseyJzdHlsZSI6eyJoZWFkIjp7Im5hbWUiOiJub25lIn19fV0sWzMsNCwiIiwwLHsic3R5bGUiOnsiaGVhZCI6eyJuYW1lIjoibm9uZSJ9fX1dLFs1LDMsIiIsMCx7InN0eWxlIjp7ImhlYWQiOnsibmFtZSI6Im5vbmUifX19XSxbNSw2LCIiLDIseyJzdHlsZSI6eyJoZWFkIjp7Im5hbWUiOiJub25lIn19fV1d
\begin{tikzcd}
	&&& \cdast \\
	&& {} \\
	&& \cdast && {V_1} \\
	\\
	& {W_1} && {W_2} \\
	\\
	\bullet && {V_2}
	\arrow[no head, from=1-4, to=3-3]
	\arrow[no head, from=1-4, to=3-5]
	\arrow[no head, from=3-3, to=5-2]
	\arrow[no head, from=3-3, to=5-4]
	\arrow[no head, from=5-2, to=7-1]
	\arrow[no head, from=5-2, to=7-3]
\end{tikzcd}
&
% https://q.uiver.app/#q=WzAsNixbMCw1LCJVXzIiXSxbMSwzLCJXXzEiXSxbMiw1LCJWXzIiXSxbMiwxLCJcXGNkYXN0Il0sWzMsMywiV18yIl0sWzIsMF0sWzEsMCwiIiwwLHsic3R5bGUiOnsiaGVhZCI6eyJuYW1lIjoibm9uZSJ9fX1dLFsxLDIsIiIsMix7InN0eWxlIjp7ImhlYWQiOnsibmFtZSI6Im5vbmUifX19XSxbMywxLCIiLDIseyJzdHlsZSI6eyJoZWFkIjp7Im5hbWUiOiJub25lIn19fV0sWzMsNCwiIiwwLHsic3R5bGUiOnsiaGVhZCI6eyJuYW1lIjoibm9uZSJ9fX1dXQ==
\begin{tikzcd}
	&& {} \\
	&& \cdast \\
	\\
	& {W_1} && {W_2} \\
	\\
	{U_2} && {V_2}
	\arrow[no head, from=2-3, to=4-2]
	\arrow[no head, from=2-3, to=4-4]
	\arrow[no head, from=4-2, to=6-1]
	\arrow[no head, from=4-2, to=6-3]
\end{tikzcd}
\\
(T_2)
&
(U_1)
\end{array}  
\end{equation}
These observations are implemented in Agda as the following:
\[
\begin{array}{ll}
  \multicolumn{2}{l}{\record \:\:  \ContainsLeft ~ (U_1 ~ U_2 : \Tree) : \Set \:\: \where} \\
  \multicolumn{2}{l}{\quad \mathsf{constructor} ~ \mathsf{gt}} \\
  \multicolumn{2}{l}{\quad \field} \\
  \;\; \quad \{W_1 ~ W_2 \} &: \Tree \\
  \;\; \quad q   &: \pathT{W_1} \\
  \;\; \quad eqT &: T_2 \equiv \sub{p_1}{(W_1 \cdast W_2)} \\
  \;\; \quad eqU &: U_1 \equiv \sub{(q \btleft W_2)}{U_2} \\
  \;\; \quad eqp &: \subst ~ Path ~ eqT ~ p_2 \equiv p_1 \append ~ (q \btleft W_2)
\end{array}
\]
Elements in this type include two implicit terms $W_1$ and $W_2$ that represent the left and right subtrees of $U_1$, respectively, along with a path $q : \pathT{W_1}$ that indicates how to extend $p_1$. Additionally, there are three equalities concerning $T_2$, $U_1$, and $p_2$.
The first and second equalities correspond to our observations from the left and right diagrams in (\ref{eq:ContainsLeft:T2U1}), respectively.
The third equality reflects what we observed in the right diagram of (\ref{eq:ContainsLeft:p1p2}).
\\
\udl{$U_1$ contains $U_2$ in its right subtree.}
\[
\begin{array}{ll}
  \multicolumn{2}{l}{\record \:\:  \ContainsRight ~ (U_1 ~ U_2 : \Tree) : \Set \:\: \where} \\
  \multicolumn{2}{l}{\quad \mathsf{constructor} ~ \mathsf{gt}} \\
  \multicolumn{2}{l}{\quad \field} \\
  \;\; \quad \{W_1 ~ W_2 \} &: \Tree \\
  \;\; \quad q   &: \pathT{W_2} \\
  \;\; \quad eqT &: T_2 \equiv \sub{p_1}{(W_1 \cdast W_2)} \\
  \;\; \quad eqU &: U_1 \equiv \sub{(W_1 \btright q)}{U_2} \\
  \;\; \quad eqp &: \subst ~ Path ~ eqT ~ p_2 \equiv p_1 \append ~ (W_1 \btright q)
\end{array}
\]
This represents the dual case of what we just examined. Notice how the triangles in the type are flipped compared to the previous case.
\\
\udl{$U_1$ and $U_2$ are disjoint.}
In this case, $p_1$ and $p_2$ diverge after a certain point. They share a common initial path, but then split at a node where they take different directions. Consider the following tree (where $V$ is an arbitrary tree):
\[
% https://q.uiver.app/#q=WzAsNSxbMiwwLCJcXGNkYXN0Il0sWzEsMiwiXFxjZGFzdCJdLFszLDIsIlYiXSxbMCw0LCJcXGJ1bGxldCJdLFsyLDQsIlxcYnVsbGV0Il0sWzAsMSwiIiwwLHsic3R5bGUiOnsiaGVhZCI6eyJuYW1lIjoibm9uZSJ9fX1dLFswLDIsIiIsMix7InN0eWxlIjp7ImhlYWQiOnsibmFtZSI6Im5vbmUifX19XSxbMCwxLCIiLDIseyJvZmZzZXQiOjEsImNvbG91ciI6WzAsNjAsNjBdLCJzdHlsZSI6eyJoZWFkIjp7Im5hbWUiOiJub25lIn19fV0sWzEsMywiIiwyLHsic3R5bGUiOnsiaGVhZCI6eyJuYW1lIjoibm9uZSJ9fX1dLFsxLDQsIiIsMix7InN0eWxlIjp7ImhlYWQiOnsibmFtZSI6Im5vbmUifX19XSxbMSw0LCIiLDIseyJvZmZzZXQiOi0xLCJjb2xvdXIiOlsyNDAsNjAsNjBdLCJzdHlsZSI6eyJoZWFkIjp7Im5hbWUiOiJub25lIn19fV0sWzEsMywiIiwyLHsib2Zmc2V0IjoxLCJjb2xvdXIiOlswLDYwLDYwXSwic3R5bGUiOnsiaGVhZCI6eyJuYW1lIjoibm9uZSJ9fX1dLFswLDEsIiIsMix7Im9mZnNldCI6LTEsImNvbG91ciI6WzI0MCw2MCw2MF0sInN0eWxlIjp7ImhlYWQiOnsibmFtZSI6Im5vbmUifX19XV0=
\begin{tikzcd}
	&& \cdast \\
	\\
	& \cdast && V \\
	\\
	\bullet && \bullet
	\arrow[no head, from=1-3, to=3-2]
	\arrow[shift right, color={rgb,255:red,214;green,92;blue,92}, no head, from=1-3, to=3-2]
	\arrow[shift left, color={rgb,255:red,92;green,92;blue,214}, no head, from=1-3, to=3-2]
	\arrow[no head, from=1-3, to=3-4]
	\arrow[no head, from=3-2, to=5-1]
	\arrow[shift right, draw={rgb,255:red,214;green,92;blue,92}, no head, from=3-2, to=5-1]
	\arrow[no head, from=3-2, to=5-3]
	\arrow[shift left, draw={rgb,255:red,92;green,92;blue,214}, no head, from=3-2, to=5-3]
\end{tikzcd}
\]
The red path represents $p_1$ while the blue path shows $p_2$ Notice how they share a common initial segment. Let us call this shared part $q$ with the remaining segments of $p_1$ and $p_2$ as $q_1$ and $q_2$ respectively. This gives us the following diagram:
\begin{equation}\label{eq:LeftRight:paths}
% https://q.uiver.app/#q=WzAsNSxbMiwwLCJcXGNkYXN0Il0sWzEsMiwiXFxjZGFzdCJdLFszLDIsIlYiXSxbMCw0LCJcXGJ1bGxldCJdLFsyLDQsIlxcYnVsbGV0Il0sWzAsMSwicSIsMix7ImNvbG91ciI6WzI3MCw2MCw2MF0sInN0eWxlIjp7ImhlYWQiOnsibmFtZSI6Im5vbmUifX19LFsyNzAsNjAsNjAsMV1dLFswLDIsIiIsMix7InN0eWxlIjp7ImhlYWQiOnsibmFtZSI6Im5vbmUifX19XSxbMSwzLCIiLDIseyJzdHlsZSI6eyJoZWFkIjp7Im5hbWUiOiJub25lIn19fV0sWzEsNCwiIiwyLHsic3R5bGUiOnsiaGVhZCI6eyJuYW1lIjoibm9uZSJ9fX1dLFsxLDQsInFfMiIsMCx7Im9mZnNldCI6LTEsImNvbG91ciI6WzI0MCw2MCw2MF0sInN0eWxlIjp7ImhlYWQiOnsibmFtZSI6Im5vbmUifX19LFsyNDAsNjAsNjAsMV1dLFsxLDMsInFfMSIsMix7Im9mZnNldCI6MSwiY29sb3VyIjpbMCw2MCw2MF0sInN0eWxlIjp7ImhlYWQiOnsibmFtZSI6Im5vbmUifX19LFswLDYwLDYwLDFdXV0=
\begin{tikzcd}
	&& \cdast \\
	\\
	& \cdast && V \\
	\\
	\bullet && \bullet
	\arrow["q"', color={rgb,255:red,153;green,92;blue,214}, no head, from=1-3, to=3-2]
	\arrow[no head, from=1-3, to=3-4]
	\arrow[no head, from=3-2, to=5-1]
	\arrow["{q_1}"', shift right, color={rgb,255:red,214;green,92;blue,92}, no head, from=3-2, to=5-1]
	\arrow[no head, from=3-2, to=5-3]
	\arrow["{q_2}", shift left, color={rgb,255:red,92;green,92;blue,214}, no head, from=3-2, to=5-3]
\end{tikzcd}  
\end{equation}
We obtain $T_1$ by substituting $U_2$ at the hole along path $q \append ~ q_2$, while similarly, we get $T_2$ by substituting $U_1$ at the hole along path $q \append ~ q_1$.
\begin{equation}\label{eq:LeftRight:trees}
  \arraycolsep=1cm
\begin{array}{cc}
% https://q.uiver.app/#q=WzAsNSxbMiwwLCJcXGNkYXN0Il0sWzEsMiwiXFxjZGFzdCJdLFszLDIsIlYiXSxbMCw0LCJcXGJ1bGxldCJdLFsyLDQsIlVfMiJdLFswLDEsInEiLDIseyJjb2xvdXIiOlsyNzAsNjAsNjBdLCJzdHlsZSI6eyJoZWFkIjp7Im5hbWUiOiJub25lIn19fSxbMjcwLDYwLDYwLDFdXSxbMCwyLCIiLDIseyJzdHlsZSI6eyJoZWFkIjp7Im5hbWUiOiJub25lIn19fV0sWzEsMywiIiwyLHsic3R5bGUiOnsiaGVhZCI6eyJuYW1lIjoibm9uZSJ9fX1dLFsxLDQsIiIsMix7InN0eWxlIjp7ImhlYWQiOnsibmFtZSI6Im5vbmUifX19XSxbMSw0LCJxXzIiLDAseyJvZmZzZXQiOi0xLCJjb2xvdXIiOlsyNDAsNjAsNjBdLCJzdHlsZSI6eyJoZWFkIjp7Im5hbWUiOiJub25lIn19fSxbMjQwLDYwLDYwLDFdXSxbMSwzLCJxXzEiLDIseyJvZmZzZXQiOjEsImNvbG91ciI6WzAsNjAsNjBdLCJzdHlsZSI6eyJoZWFkIjp7Im5hbWUiOiJub25lIn19fSxbMCw2MCw2MCwxXV1d
\begin{tikzcd}
	&& \cdast \\
	\\
	& \cdast && V \\
	\\
	\bullet && {U_2}
	\arrow["q"', color={rgb,255:red,153;green,92;blue,214}, no head, from=1-3, to=3-2]
	\arrow[no head, from=1-3, to=3-4]
	\arrow[no head, from=3-2, to=5-1]
	\arrow["{q_1}"', shift right, color={rgb,255:red,214;green,92;blue,92}, no head, from=3-2, to=5-1]
	\arrow[no head, from=3-2, to=5-3]
	\arrow["{q_2}", shift left, color={rgb,255:red,92;green,92;blue,214}, no head, from=3-2, to=5-3]
\end{tikzcd}
&
% https://q.uiver.app/#q=WzAsNSxbMiwwLCJcXGNkYXN0Il0sWzEsMiwiXFxjZGFzdCJdLFszLDIsIlYiXSxbMCw0LCJVXzEiXSxbMiw0LCJcXGJ1bGxldCJdLFswLDEsInEiLDIseyJjb2xvdXIiOlsyNzAsNjAsNjBdLCJzdHlsZSI6eyJoZWFkIjp7Im5hbWUiOiJub25lIn19fSxbMjcwLDYwLDYwLDFdXSxbMCwyLCIiLDIseyJzdHlsZSI6eyJoZWFkIjp7Im5hbWUiOiJub25lIn19fV0sWzEsMywiIiwyLHsic3R5bGUiOnsiaGVhZCI6eyJuYW1lIjoibm9uZSJ9fX1dLFsxLDQsIiIsMix7InN0eWxlIjp7ImhlYWQiOnsibmFtZSI6Im5vbmUifX19XSxbMSw0LCJxXzIiLDAseyJvZmZzZXQiOi0xLCJjb2xvdXIiOlsyNDAsNjAsNjBdLCJzdHlsZSI6eyJoZWFkIjp7Im5hbWUiOiJub25lIn19fSxbMjQwLDYwLDYwLDFdXSxbMSwzLCJxXzEiLDIseyJvZmZzZXQiOjEsImNvbG91ciI6WzAsNjAsNjBdLCJzdHlsZSI6eyJoZWFkIjp7Im5hbWUiOiJub25lIn19fSxbMCw2MCw2MCwxXV1d
\begin{tikzcd}
	&& \cdast \\
	\\
	& \cdast && V \\
	\\
	{U_1} && \bullet
	\arrow["q"', color={rgb,255:red,153;green,92;blue,214}, no head, from=1-3, to=3-2]
	\arrow[no head, from=1-3, to=3-4]
	\arrow[no head, from=3-2, to=5-1]
	\arrow["{q_1}"', shift right, color={rgb,255:red,214;green,92;blue,92}, no head, from=3-2, to=5-1]
	\arrow[no head, from=3-2, to=5-3]
	\arrow["{q_2}", shift left, color={rgb,255:red,92;green,92;blue,214}, no head, from=3-2, to=5-3]
\end{tikzcd}
\\
(T_1)
&
(T_2)
\end{array}
\end{equation}
We implement the observations in Agda as the following type:
\[
\begin{array}{ll}
  \multicolumn{2}{l}{\record \:\:  \LeftRight ~ (U_1 ~ U_2 : \Tree) : \Set \:\: \where} \\
  \multicolumn{2}{l}{\quad \mathsf{constructor} ~ \mathsf{disj}} \\
  \multicolumn{2}{l}{\quad \field} \\
  \;\; \quad \{W ~ W_1 ~ W_2 \} &: \Tree \\
  \;\; \quad q       &: \pathT{W} \\
  \;\; \quad q_1     &: \pathT{W_1} \\
  \;\; \quad q_2     &: \pathT{W_2} \\
  \;\; \quad eqT_1   &: \sub{q}{(W_1 \cdast \sub{q_2}{U_2})} \equiv T_1 \\
  \;\; \quad eqT_2   &: T_2 \equiv \sub{q}{(\sub{q_1}{U_1} \cdast W_2)} \\
  \;\; \quad eqp_1   &: \subst ~ Path ~ eqT_1 ~ (q \append ~ (q_1 \btleft W_2)) \equiv p_1 \\
  \;\; \quad eqp_2   &: \subst ~ Path ~ eqT_2 ~ p_2 \equiv q \append ~ (W_1 \btright q_2)
\end{array}
\]
Elements of this type include three implicit terms: $W$, $W_1$, and $W_2$. Here, $W$ represents the structure containing both the common path $q$ and the largest subtree beginning at the divergence point of $p_1$ and $p_2$. Meanwhile, $W_1$ represents the subtree where $U_1$ will be substituted, and $W_2$ represents the subtree where $U_2$ will be substituted.
The remaining equalities reflect what we observed in the diagrams shown in (\ref{eq:LeftRight:paths}) and (\ref{eq:LeftRight:trees}).

While our illustration shows $U_1$ positioned to the left of $U_2$, the dual case where $U_2$ appears to the left of $U_1$ is also possible and is covered by this type definition.

After having distinguished the cases above, we define a datatype to collect all possible cases of the relative positions of $U_1$ and $U_2$.
In the types above, we do not care about the order of $U_1$ and $U_2$ and for a better proving experience (using the datatype in the subsequent proofs), we take the order of $U_1$ and $U_2$ into account and develop seven cases.

\[
\begin{array}{rl}
  \multicolumn{2}{l}{\data \:\:  \mathsf{SubEqCases} \:\: (U_1 ~ U_2 : Tree) : \Set \:\: \where} \\
  \;\; \oneeqtwo &: \Same ~ p_1 ~ p_2 ~ U_1 ~ U_2 \to \subcases ~ U_1 ~ U_2 \\
  \;\; \twogtLone &: \ContainsLeft ~ p_1 ~ p_2 ~ U_1 ~ U_2 \to \subcases ~ U_1 ~ U_2 \\
  \;\; \twogtRone &: \ContainsRight ~ p_1 ~ p_2 ~ U_1 ~ U_2 \to \subcases ~ U_1 ~ U_2 \\
  \;\; \onegtLtwo &: \ContainsLeft ~ p_2 ~ p_1 ~ U_2 ~ U_1 \to \subcases ~ U_1 ~ U_2 \\
  \;\; \onegtRtwo &: \ContainsRight ~ p_2 ~ p_1 ~ U_2 ~ U_1 \to \subcases ~ U_1 ~ U_2 \\
  \;\; \oneLtwoR &: \LeftRight ~ p_1 ~ p_2 ~ U_1 ~ U_2 \to \subcases ~ U_1 ~ U_2 \\
  \;\; \oneRtwoL &: \LeftRight ~ p_2 ~ p_1 ~ U_2 ~ U_1 \to \subcases ~ U_1 ~ U_2
\end{array}
\]
The constructor $\onegtLtwo$ means $U_2$ is inside $U_1$ and the path $p_2$ is longer than $p_1$.
The other constructors are read in a similar manner, but one should be careful about the order of arguments. For example, in the constructor $\twogtLone$, the arguments are swapped.

In the final step, we construct a function to examine the correctness of the types we have defined above.
\[
\begin{array}{l}
 \subeq : \forall ~ \{T_1 ~ T_2 \} ~ U_1 ~ U_2 ~ (p_1 : \pathT{T_1}) ~ (p_2 : \pathT{T_2})
  \\
  \quad \to \sub{p_1}{U_1} \equiv \sub{p_2}{U_2}
  \\
  \quad \to \subcases ~ p_1 ~ p_2 ~ U_1 ~ U_2
\end{array}
\]
This function demonstrates that when two substitutions are equal, we can always identify a corresponding type from one of the cases defined above. The proof proceeds by pattern-matching on paths $p_1$ and $p_2$. For simple base cases, the proof directly constructs the appropriate case value. For recursive cases, we use Agda's $\mathsf{with}$ to perform intermediate computations and use the intermediate results to construct the desired term in the corresponding type.

\section{Nonassociative Lambek calculus}\label{sec:calculus}
In this section, we present the sequent calculus for the nonassociative Lambek calculus and its cut-elimination property.

A sequent takes the form $T \vd C$, where $T$ is a tree and $C$ is a single formula.
The derivation rules of the calculus are given below:
\begin{equation}\label{eq:seqcalc}
\begin{array}{c}
  \infer[\ax]{X \vd X}{}
  \quad
  \infer[\Rr]{T \vd A \Rarr B}{A , T \vd B} 
  \quad
  \infer[\Lr]{T \vd A \Rarr B}{T , A \vd B \Larr A} 
  \\[7pt]
  \infer[\Rl]{T[U , A \Rarr B] \vd C}{
    U \vd A
    &
    T[B] \vd C
  }
  \quad
  \infer[\Ll]{T[U , B \Larr A] \vd C}{
    U \vd A
    &
    T[B] \vd C
  }
  \\[7pt]
  \infer[\tr]{T , U \vd A \ot B}{
    T \vd A
    &
    U \vd B
  }
  \quad
  \infer[\tl]{T[A \ot B] \vd C}{
    T[A , B] \vd C
  }
\end{array}
\end{equation}

In our Agda implementation, derivations are encoded as an inductive datatype where each rule in (\ref{eq:seqcalc}) corresponds to exactly one constructor. Here we illustrate two examples, $\ax$ and $\Rl$:
\[
\begin{array}{l}
\ax : \forall \{X\} 
\\
\quad \to \eta ~ (\at ~ X) \vd \at ~ X
\end{array}
\]

This constructor specifies that for any atomic formula $X : \At$, there exists a term of type $\eta ~ (\at ~ X) \vd \at ~ X$, directly corresponding to the $\ax$ rule in (\ref{eq:seqcalc}). Since the antecedent is a tree without holes, this construction avoids introducing inconsistent derivations.

The $\Rl$ rule is implemented as:
\[
\begin{array}{rcl}
\Rl & : & \forall ~ \{T ~ U ~ A ~ B ~ C \} \\
&\to& (p : \pathT{T}) \\
&\to&(f : U \vd A) ~ (g : \sub{p}{(\eta ~ B)} \vd C) \\
&\to& \sub{p}{(U\cdast ~ \eta ~ (A \Rarr B))} \vd C
\end{array}
\]

Thus, given terms $f : U \vd A$ and $g : \sub{p}{(\eta ~ B)} \vd C$, the expression $\Rl ~f ~g$ produces a term of type $\sub{p}{(U\cdast ~ \eta ~ (A \Rarr B))} \vd C$.

The admissible cut rule for this calculus takes the form:
\[
\begin{array}{c}
  \infer[\cut]{T[U] \vd C}{
    U \vd D
    &
    T[D] \vd C
  }
\end{array}
\]

In Agda, we implement the cut rule as:
\[
\begin{array}{l}
  \cut : (f : U \vd D) ~(g : W \vd C) 
  \\
  \quad \to (eq : W \equiv \sub{p}{(\eta ~ D)})
  \\
  \quad \to \sub{p}{U} \vd C
\end{array}
\]

We introduce an auxiliary term $eq$ to enable Agda to pattern-match on $g$.
The proof proceeds by induction on $g$. Similar to pen-and-paper proofs, in certain cases we need to determine the relative positions of the cut formula and the principal formula. For instance, when $g = \tl g'$, the derivation appears as:
\[
\begin{array}{c}
  \infer[\cut]{T[U] \vd C}{
    \deduce{U \vd D}{f}
    &
    \infer[\tl]{T'[A \ot B] \vd C}{
      \deduce{T'[A , B] \vd C}{g'}
    }
  }
\end{array}
\] 

Here, the tree $T'[A\ot B]$ must have another representation as $T''[D]$ for some tree $T''$. In pen-and-paper proofs, we need to determine whether $D$ exactly matches the principal formula $A \ot B$.

In our Agda implementation, we use the $\mf{with}$ construct and the $\subeq$ function to perform intermediate computation and identify the different possible cases.
There are three subcases: $\oneeqtwo$ (where $D$ is exactly the principal formula $A \ot B$), $\oneLtwoR$ (where $D$ and $A \ot B$ are in different subtrees with $A \ot B$ on the left and $D$ on the right), and $\oneRtwoL$ (the dual case).
For the latter two cases, we permute the $\tl$ rule downward and continue recursively.
For the exact match case, we introduce an auxiliary function:
\[
\begin{array}{l}
  \cut\tl : (f : U \vd A \ot B) ~(g : \sub{p}{(\eta ~ A \cdast \eta ~ B)} \vd C) 
  \\
  \quad \to \sub{p}{U} \vd C
\end{array}
\]

This function is proven by pattern-matching on $f$, mirroring the pen-and-paper approach where we further induct on the other derivation when the cut formula and principal formula coincide.
We employ this auxiliary function to avoid termination checking failures in subsequent proofs involving properties of $\cut$.
The same technique of using auxiliary functions is also applied to the rules $\Rl$ and $\Ll$.
\begin{remark}
  An alternative approach would be to prove $\cut$ by first pattern-matching on $f$ and distinguishing cases where $f$ concludes with a right rule. However, this approach results in a substantially longer proof that requires numerous applications of the $\mf{with}$ construct, significantly increasing type-checking time. 
\end{remark}

\section{Equivalence of derivations}

\section{Properties of Cut}

\section{Interpolation}

\section{Concluding Remarks}

\begin{credits}
\subsubsection{\ackname} This work was supported by the Estonian Research Council grant PSG749. 

% \subsubsection{\discintname}
% It is now necessary to declare any competing interests or to specifically
% state that the authors have no competing interests. Please place the
% statement with a bold run-in heading in small font size beneath the
% (optional) acknowledgments\footnote{If EquinOCS, our proceedings submission
% system, is used, then the disclaimer can be provided directly in the system.},
% for example: The authors have no competing interests to declare that are
% relevant to the content of this article. Or: Author A has received research
% grants from Company W. Author B has received a speaker honorarium from
% Company X and owns stock in Company Y. Author C is a member of committee Z.
\end{credits}
%
% ---- Bibliography ----
%
% BibTeX users should specify bibliography style 'splncs04'.
% References will then be sorted and formatted in the correct style.
%
\bibliographystyle{splncs04}
\bibliography{tableaux}
%
\end{document}
